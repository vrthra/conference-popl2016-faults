\documentclass[preprint,nonatbib]{sigplanconf}\usepackage[]{graphicx}\usepackage[]{color}
%% maxwidth is the original width if it is less than linewidth
%% otherwise use linewidth (to make sure the graphics do not exceed the margin)
\makeatletter
\def\maxwidth{ %
  \ifdim\Gin@nat@width>\linewidth
    \linewidth
  \else
    \Gin@nat@width
  \fi
}
\makeatother

\usepackage{Sweavel}


%\usepackage{cite}
\usepackage{amssymb}
\usepackage{amsmath}
\newcommand{\cL}{{\cal L}}
\usepackage{placeins}
\usepackage{xspace}
\usepackage{boxedminipage}
\usepackage{tikz}

\usepackage{algpseudocode}
\usepackage{algorithm}
\usepackage{algorithmicx}
\newcommand{\theHalgorithm}{\arabic{algorithm}}

\usepackage[bookmarks=false, urlcolor=blue, linkcolor=blue, citecolor=blue, colorlinks=true]{hyperref} %\usepackage[center]{caption}
%\linespread{2}
\usepackage{comment}

\usepackage{listings}
\usepackage{float}
\floatstyle{plain}
\newfloat{program}{t}{lop}
\floatname{program}{Figure}
\usepackage{color}

\definecolor{dkgreen}{rgb}{0,0.6,0}
\definecolor{gray}{rgb}{0.5,0.5,0.5}
\definecolor{mauve}{rgb}{0.58,0,0.82}
\definecolor{lightgray}{gray}{0.75}

\newcommand\greybox[1]{%
  \vskip\baselineskip%
  \par\noindent\colorbox{lightgray}{%
    \begin{minipage}{0.99\columnwidth}#1\end{minipage}%
  }%
  \vskip\baselineskip%
}

\usepackage{blindtext}
\usepackage{framed}

\floatstyle{boxed}
\newfloat{infobox}{tbp}{ext}
\restylefloat*{infobox}
\floatname{infobox}{Note}

\floatstyle{ruled}
%\floatstyle{boxed}
\newfloat{mybox}{thp}{lop}
\floatname{mybox}{Box}

\newcommand{\faultT}{composite fault hypothesis\xspace}
\newcommand{\FaultT}{Composite fault hypothesis\xspace}
\newcommand{\efaultT}{\emph{\faultT}\xspace} %          composite fault hypothesis
\newcommand{\eFaultT}{\emph{\FaultT}\xspace} %          Composite fault hypothesis

\newcommand{\couplingC}{composite coupling\xspace}%          composite coupling
\newcommand{\CouplingC}{Composite coupling\xspace}%          Composite coupling
\newcommand{\kappaT}{\couplingC ratio\xspace}%          composite coupling ratio
\newcommand{\KappaT}{\CouplingC ratio\xspace}%          Composite coupling ratio

\newcommand{\thece}{the \emph{coupling effect}\xspace}
\newcommand{\Thece}{The \emph{coupling effect}\xspace}

\newcommand{\finput}{\emph{domain}\xspace}
\newcommand{\fInput}{\emph{Domain}\xspace}
\newcommand{\foutput}{\emph{co-domain}\xspace}
\newcommand{\fOutput}{\emph{Co-Domain}\xspace}

\newcommand{\oHypothesis} {
complex faults are coupled to simple faults in such
a way that a test data set that detects all simple faults in a program
will detect a high percentage of the complex faults.}

\newcommand{\cHypothesis}{
  Tests detecting a fault in isolation will (with high probability $\kappa$) continue to
detect the fault even when it occurs in combination with
other faults.\xspace
}%


\newcommand{\lcalc}{$\lambda$-calculus\xspace}
\newcommand{\clogic}{Combinatory-logic\xspace}
\newcommand{\qfunction}{\emph{q}-function\xspace}
\newcommand{\qfunctions}{\emph{q}-functions\xspace}

\newcommand*\per{\scalebox{.5}{\%}}



\begin{document}
\special{papersize=8.5in,11in}
\setlength{\pdfpageheight}{\paperheight}
\setlength{\pdfpagewidth}{\paperwidth}

\conferenceinfo{CONF 'yy}{Month d--d, 20yy, City, ST, Country}
\copyrightyear{2017}
\copyrightdata{978-1-nnnn-nnnn-n/yy/mm}
\copyrightdoi{nnnnnnn.nnnnnnn}

\def\mytitle{The Theory of Composite Faults}
%\def\mytitle{The Semiotics of Fault Masking}

% \conferenceinfo{CONF 'yy}{Month d--d, 20yy, City, ST, Country}
% \copyrightyear{20yy}
% \copyrightdata{978-1-nnnn-nnnn-n/yy/mm}
% \copyrightdoi{nnnnnnn.nnnnnnn}

% Uncomment the publication rights you want to use.
%\publicationrights{transferred}
%\publicationrights{licensed}     % this is the default
%\publicationrights{author-pays}

% \titlebanner{banner above paper title}        % These are ignored unless
% \preprintfooter{short description of paper}   % 'preprint' option specified.

\title{\mytitle}
%\subtitle{Subtitle Text, if any}

\authorinfo{}{}{}
%\authorinfo{Rahul Gopinath\and Carlos Jensen\and Alex Groce}
%  	       {Oregon State University}
%  	       {\{gopinatr,cjensen\}@eecs.orst.edu, agroce@gmail.com}

\hypersetup{
    pdftitle = {\mytitle},
%    pdfauthor = {Rahul Gopinath}
}

\maketitle

\begin{abstract}

Fault masking is an important problem in software testing, where the
effect of one fault serves to mask that of another fault for particular
test inputs.
The \emph{coupling effect} is relied upon by testing practitioners to
ensure that fault masking is rare.
It states that \emph{\oHypothesis}.

While this effect has been empirically evaluated, our theoretical
understanding of \emph{the coupling effect} is as yet incomplete.
Wah proposed a theory of the coupling effect on finite bijective (or near bijective)
functions with the same \finput and \foutput, and assuming uniform distribution
for candidate functions. This model however, was criticized as being too
simple to model real systems, as it did not account for differing \finput and
\foutput in real programs,
% the impact of recursion and iteration,
or for syntactic neighborhood (it assumed all faults equally likely).

We propose a new theory of fault coupling that is applicable in the realm of
general functions (with certain constraints). We show that there are two kinds
of fault interactions, of which only the weak interaction can be modeled by
the theory of the coupling effect. The strong interaction can produce faults that
are semantically different from the original faults. These faults should
hence be considered as independent atomic faults.
Our analysis show that the theory
holds even when the effect of syntactical neighborhood of the program is
considered.

We also propose a modified hypothesis that is stronger than the traditional
definition --- the \emph{\faultT}:

\textit{\cHypothesis}

We analyze numerous real-world programs with real faults to validate
and empirically approximate the \emph{composite coupling ratio} $\kappa$.
We find that $\kappa$ is approximately $99\%$ with
95\% confidence, and very close to the general coupling ratio $C$
which was also found to be greater than $99\%$ with 95\% confidence.

\end{abstract}

\category{D}{2}{4}

\begin{comment}
\begin{CCSXML}
  <ccs2012>
  <concept>
  <concept_id>10011007.10011074.10011099.10011693</concept_id>
  <concept_desc>Software and its engineering~Empirical software validation</concept_desc>
  <concept_significance>500</concept_significance>
  </concept>
  </ccs2012>
\end{CCSXML}

\ccsdesc[500]{Software and its engineering~Empirical software validation}
\end{comment}

% general terms are not compulsory anymore,
% you may leave them out
\terms
faults, coupling

\keywords
test frameworks, evaluation of coverage, coverage criteria, statistical analysis


% 10 (include figures and appendices but not reference) + 1 (references)
% we investigate another dimension vis patterns.
\begin{comment}

\end{comment}


\setlength\tabcolsep{2pt}


% 10 pages for the main text

\section{Introduction}

\emph{Fault masking} occurs when interactions between component faults in
a complex fault result in expected (non-faulty) value being produced for
particular test inputs. This can result in faults being missed by test cases,
even though the complex fault may produce faulty outputs for other values, hence
negatively impacting the testability of a software system.
It can also result in undeserved overconfidence in the reliability of a software
system.
%Fault masking also negatively impacts the testability of a software
%system, making bugs hard to find.

The \emph{coupling effect}~\cite{demillo1978hints} hypothesis studies the
semiotics\footnote{The relation between syntax and semantics of faults.} of fault masking. It asserts that
\emph{``complex faults are coupled to simple faults in such a way that a test
data set that detects \textbf{all simple faults in a program} will detect a
high percentage of the complex faults.''}~\cite{offutt1992investigations,offutt1989thecoupling, jia2011analysis}.

This is relied upon by software testers to assert that fault masking is indeed
rare. Further, Mutation analysis~\cite{lipton1971fault,budd1979mutation}, the
premier technique for evaluating test suite quality relies on the \emph{coupling
effect} to assert that detecting simple faults is sufficient to guarantee the
quality of a test suite against more complex faults.

However, our understanding of the \emph{coupling effect} is woefully
inadequate. We do not know when, and how often fault coupling can happen,
whether multiple faults will always result in fault coupling, and the effect of
increase in number of faults in the number of faults masked.

Further, the formal statement of coupling effect itself is ambiguous and
inadequate as it covers only the case where all simple faults have been found.
Even worse, it has no unambiguous definition of what a simple (or atomic) fault
is. We propose a stronger version of the \emph{coupling effect}
(called the \efaultT to avoid confusion):

\textit{
  \cHypothesis
}

% With a corresponding change in mutation analysis to exhaustively generate all
% atomic faults (rather than simple faults), \efaultT can be used in an inductive
% argument to assert the validity of mutation analysis.

We investigate our hypothesis both theoretically and empirically. We describe
how faults interact, categorize fault interaction to weak and strong
interaction, and provide an unambiguous description of atomic faults. We further
validate our findings using real faults from numerous software systems.
The terms used in this paper are given in Note~\ref{box:terms}.


\begin{infobox}
\noindent\emph{(Semantic) Separability of faults}: Two faults present in a function are said to be separable \emph{if and only if} the smallest possible chunk containing both faults can be decomposed into two functions $g$ and $h$ such that each fault is isolated within a single function (providing $g_a$ and $h_b$ as faulty functions), the behavior of composition $h \circ g$ equals the behavior of the original chunk in terms of input and output , and composition $h_b \circ g_a$ equals the behavior of the chunk with both faults. \\
\noindent\emph{Simple fault}: A fault that cannot be \emph{lexically} separated into other independent smaller faults. Also called a \emph{first order} fault. \\
\noindent\emph{Complex fault}: A fault that can be \emph{lexically} separated into independent smaller faults. Also called a \emph{higher order} fault, or a \emph{combined fault}. \\
\noindent\emph{Constituent fault}: A fault that is \emph{lexically} contained in another fault.\\
\noindent\emph{Atomic fault}: A fault that cannot be \emph{semantically} separated into other independent smaller faults.\\
\noindent\emph{Composite fault}: A fault that can be \emph{semantically} separated into independent smaller faults. \\
\noindent\emph{Traditional coupling ratio ($C$)}: The ratio between the percentage of complex faults detected and the percentage of simple faults that were detected by a test suite.\\
\noindent\emph{\KappaT ($\kappa$)}: The ratio between the percentage of complex faults detected by the same set of test cases that detected the constituent simple faults, and the percentage of constituent simple faults detected.\\
% DONE: clarify what happens when the \finput of the mutant is different, or add it to assumptions
% WONTFIX: e.g. for a constant program that does not take input.
\noindent\emph{\fInput of a function}: The \finput of a function is the set of all values a function can take as inputs (this is practically the input type of a function).\\
% DONE: clarify what happens when the \foutput of the mutant is different, or add it to assumptions
% DONE: e.g when multiple elements in \finput maps to the same \foutput (use co-domain rather than range/image)
\noindent\emph{\fOutput of a function}: The \foutput of a function is the set of all values that a function can produce when it is provided with a valid input from its \finput (this is practically the output type of a function).\\
\noindent\emph{Range of a function}: The \emph{range} of a function is the set of all values in \foutput that directly maps to a value in the \finput.\\
\noindent\emph{Finite \finput and \foutput}: We assume that the set of possible input values, and the set of output values corresponding to them, are finite.\\
%TODO: explain dynamic languages: They have an almost infinite domain and co-domain.
\noindent\emph{Total functions}: Functions where an output is defined for all elements in its \finput.\\
\noindent\emph{Syntactic neighborhood}: The set of functions that can be reached from a given function by modifying its \emph{syntactical representation in a given language} a given number of times.\\
\noindent\emph{Degenerate function}: A function which is not \emph{bijective} from \emph{domain} to \emph{range}. If $n$ values in \emph{domain} maps to duplicate values in \emph{range}, it is called an $n$-degenerate function.
\caption{Terms used in this paper}
\label{box:terms}
\end{infobox}
\subsection{Theory}

Theoretical aspects of the coupling effect
%  for logical faults was investigated by Kapoor~\cite{kapoor2006formal}, and
on functions was investigated by Wah et al.~\cite{wah1995fault,wah2000atheoretical,wah2003ananalysis}. Wah
assumes that any software is built by composition of \emph{q} independent functions, with
a few restrictions. This model is called the standard \qfunction model.
The \qfunctions have the following restrictions:
\begin{itemize}
  \item Functions have same \finput and \emph{range} (order \emph{n}) so that any two
    functions can be composed together, and the functions are \emph{bijective} (one-to-one).
    The non-\emph{bijective} functions are modeled as \emph{degenerate} functions which
    are close to \emph{bijective}, but with a few duplicate inputs that map to
    same location in image.
  \item A program with two faults can be split into two programs, with one program
    containing one fault, and the other program the other\footnote{
      This was assumed by Wah to be true for general functions. However, we
      show in Section~\ref{sec:theory} that it is not applicable to
      all functions.
    }.
  \item All applicable functions have equal probability of being chosen as a faulty
       representation of another (ignoring the
       syntactical neighborhood of a function) -- the \emph{democratic assumption}.
  \item The number of functions considered \emph{q} is much smaller than the size of the domain. That is, $q \ll n$.
     Wah suggests that as $q$ nears $n$, the coupling effect weakens.
\end{itemize}
Wah showed that for \emph{q} functions, the survival ratio of first and second
order test sets are $\frac{1}{n}$ and $\frac{1}{n^2}$ respectively,  where $n$ is
the order of the \finput. %~\cite{jia2011analysis}.
Further, Wah finds a general observation, and a heuristic, that the survival
ratio of a multi-fault alternate is $\frac{p+1}{n}$ if there are $p$ fault free
functions at the end after a faulty function. This allows Wah to solve the
\qfunction model since there are $2^{p-1}-1$ multi fault alternates with last
faulty function at $p$. Hence, the expected number of survivors for a \qfunction
composition is given by:
\[
  \frac{1}{n^r} \sum_{p=1}^{q} (2^{p-1} -1) (q - p + 1)^r
\]
for test sets of order $r$, \qfunctions, and $n$ the order of the domain.

Wah's analysis, however, lacks wider applicability due to the constraints
placed on functions. Functions in typical programs vary widely in their \finput
and \foutput. Second, the assumption that distribution of mathematical functions
can represent the distribution of programs with same \finput and \foutput
is rather strong because there can be an infinite number of alternative programs
that represent the same mathematical function.
Third, the assumption that all applicable functions are equally
probable as faulty representation of a function ignores the
impact of syntactical neighborhood, because functions that are in the
syntactical neighborhood (those functions that can be reached by a small
modification of the original function) are more probable as faulty replacements
than others further away. That is, it is possible that a \emph{quick sort}
implementation can have a small bug, resulting in an incorrect sort. However,
it is quite improbable that it is replaced by an algorithm for --- say ---
\emph{random shuffle}, which has
the same \finput and \foutput as that of a sorting function.  While
syntactical nearness does not completely capture semantic nearness, it
is closer than assuming any function is a plausible fault for any
other function.  Next, Wah assumed that all complex faults can be semantically
separated to component faults with the component faults appearing in independent
functions. However, as we show in
Section~\ref{sec:theory}, this assumption is valid only in certain cases.
Further, Wah's analysis does not account for recursion and iteration. If one
assumes that different functions in the \qfunction model may be considered as
different steps in an iteration, it falls short due to the specification that
$q$ be much less than the domain of the function considered (at which point,
Wah suggests that the approximations no longer hold true).
Finally, Wah's analysis suggested that
the survival ratio of mutants is dependent on the \emph{domain} of the
function. However, this result was due to the assumption that both \emph{domain}
and \emph{range} of the functions examined were similar. We show that the
survival ratio of a mutant is actually dependent on the \foutput of the function
examined, when \finput and \foutput are different (we note that the size of \foutput is
bounded by the size of \finput of the function).

We propose a simpler theory of fault coupling that uses a similar model as that
of Wah, but with relaxed constraints. Our
analysis incorporates functions with differing \finput and \foutput. We clarify
the semantic separability of complex faults, and show how it affects the
coupling effect.
More importantly, we show that certain common classes of complex faults may not
semantically separable.
This provides us with a definition of an \emph{atomic fault}, as a fault that
cannot be semantically separated into simpler faults. This is important because
two faults that may be lexically separate but inseparable can be expected to
produce a different faulty behavior than either fault considered independently.
Further, we consider the impact of syntactic neighborhood. Using both case
analysis and statistical argument, we show that our
analysis remains valid even when the syntactic neighborhood is considered.

\subsection{Empirical Validation}

The first empirical study of the coupling effect was conducted by Lipton et al.~\cite{lipton1978the,demillo1978hints}.
They used a sample of $22,000$ mutants for the program \emph{find} up to the fourth order, and
observed that test cases for first order mutants were adequate for samples up
to fourth order.  More empirical evidence for the coupling effect was supplied by
Offutt~\cite{offutt1992investigations,offutt1989thecoupling} using two more
programs, \emph{mid} and \emph{tryp}, who observed that the tests for first
order mutants were sufficient to kill up to 99\% of all $2^{nd}$ order mutants, and
99\% of $3^{rd}$ order mutants sampled.

The validity of mutation analysis itself --- that mutations are coupled to
real faults --- has been demonstrated many times in the past.
DeMillo et al.~\cite{demillo1991on} showed that \TeX{} errors could be coupled to
simple mutants, and Daran et al.~\cite{daran1996software} found that 85\% of
error traces produced by mutants were similar to those produced by real faults.
Andrews et al.~\cite{andrews2005mutation,andrews2006using},
found that faults generated by mutation analysis are similar in detectability
to real world faults (in contrast to hand seeded
faults). Do et al.~\cite{do2006on} showed that using mutation faults for test
prioritization result in higher detection rates than using hand seeded faults.
Comparing four adequacy
criteria --- mutation, edge pair, all uses, and prime path ---
Li et al.~\cite{li2009experimental} showed that
mutation adequate testing could detect hand seeded faults better (85\%)
than other criteria (65\%).
Finally, Just et al.~\cite{just2014mutants} empirically showed that the
effectiveness of a test suite in detecting mutants mirrors its ability to
detect real faults, and the faults introduced by mutation were coupled to 73\%
of real faults.

We note that, as Offutt suggests~\cite{offutt1992investigations,offutt1989thecoupling},
there are two distinct definitions of coupling involved. The
\emph{general coupling effect} suggests that simple faults are coupled to more
complex faults such that test data adequate for simple faults will be able to
kill a majority of more complex faults. The \emph{mutation coupling effect}
states that test data adequate for simple first order mutants will be able to
detect a majority of more complex mutants. While the \emph{mutation
coupling effect} has been demonstrated, the \emph{general coupling
effect} has not been empirically validated yet.

Our empirical analysis aims to accomplish the following goals: First, we
empirically evaluate the value of the \couplingC ratio $\kappa$ for numerous real-world projects.
This gives us confidence in the assumptions made in the theoretical analysis,
and serves to validate the \efaultT. Second, we empirically
evaluate the general coupling effect for faults, and compute the traditional coupling ratio $C$.

What is the relation between the \couplingC ratio $\kappa$ and the traditional coupling ratio $C$?
We can regard the \kappaT as a lower limit of the traditional
coupling ratio. As we explain further, the general coupling ratio does not
discount the effect of strong fault interactions, which can produce
complex faults semantically independent from the constituent faults.
% That is, the combined faults are sometimes detected by test cases other
% than the original test cases that detected the constituent
% faults.
Hence, $C$ is not bounded by any number, and will often be larger than
$\kappa$, which will be strictly less than one.
\\

\noindent\textbf{Contributions:}
%The contributions of this paper are:
\begin{itemize}
\item We identify the vagueness of formal statement of the \emph{coupling effect} in
dealing with non-adequate mutation scores,
 and propose the \efaultT to resolve this inadequacy.
\item Our theoretical analysis results in the \efaultT for
  general functions. We find the \couplingC ratio to be $1-\frac{1}{n}$,
  where $n$ is the \foutput of function considered.
\item We show that our analysis remains valid even when considering recursion
  and loops --- common features in programming.
\item We provide empirical validation for the \efaultT, and compute the
  \couplingC ratio $\kappa$ to be greater than $0.99$, with $95\%$ confidence using 25 projects.
  This helps substantiate the impact of \couplingC.
\item We provide the first empirical validation of the \emph{general coupling
  effect}, which states that simple faults and more complex faults are
coupled.
\end{itemize}

\begin{comment}
The \efaultT also has an immediate application. We
have shown previously~\cite{gopinath2015howhard} how sampling can be used to
determine to any required precision whether a given mutant is equivalent or
not. However, the suggested method of evaluating each possibly equivalent
mutant for the full input \finput is still computationally expensive. We suggest
that all possible equivalent mutants for a particular function be combined into
a complex mutant, and evaluated on the input \finput first. If it passes all
given tests, we take all the individual mutations to be equivalent\footnote{It is
possible that they are not. However, the variants~\cite{gopinath2016measuring} produced are likely to
be very similar --- it is a compromise between accuracy and speed.}. If it
fails, split the changes equally to two parts, and recurse. Any time a set of
mutations do not fail for the given test data, the constituent mutations can be
eliminated as possibly equivalent. If a complex mutant is detected, we recurse on its
two halves until we reach the leaves. Since this is a binary tree, the worst
performance may happen when there are at least 50\% non-equivalent mutants,
where we may have to evaluate up to $2\times n$ mutants where $n$ is the total number
of mutants we need to check for equivalence.

Note that we only sample the higher order faults for validation. However, as
we showed previously~\cite{gopinath2015howhard}, sampling is a reasonable
approach as far as mutation analysis is concerned.
\end{comment}

Our full data set is available for replication\footnote{
  blinded for review.
%\url{http://eecs.osuosl.org/rahul/fse2016/}
}.
% 1. Describe the problem
% 2. State your contributions


\section{Related Work}
\label{sec:related}

Fault masking in digital circuits was studied before it was studied in
software. Dias~\cite{dias1975fault} studies the problem of fault masking, and
derives an algebraic expression that details the number of faults to be
considered for detection of all multiple faults.

Coupling effect was first studied under the aegis of \emph{mutation analysis}.
Hence we review the general mutation analysis research first.
The idea of mutation analysis was first proposed by 
Lipton~\cite{lipton1971fault}, and its main concepts were formalized by
DeMillo et al. in the ``Hints''~\cite{demillo1978hints} paper. The first
implementation of mutation analysis was provided in the PhD thesis of
Budd~\cite{budd1980theoretical} in 1980.
Previous research in mutation
analysis~\cite{budd1980mutation,mathur1994empirical,offutt1996subsumption}
suggests that it subsumes different coverage measures, including
\textit{statement}, \textit{branch}, and \textit{all-defs} dataflow
coverage~\cite{budd1980mutation,mathur1994empirical,offutt1996subsumption}.
There is also some evidence that the faults produced
by mutation analysis are similar to real faults in terms of
error trace produced~\cite{daran1996software} and the ease of
detection~\cite{andrews2005mutation,andrews2006using}.  Recent research by
Just et al.~\cite{just2014mutants} using 357 real bugs suggests that the
mutation score increases with test effectiveness for 75\% of the cases,
which was better than the 46\% reported for structural coverage.

The validity of mutation analysis rests upon two fundamental assumptions:
\emph{the competent programmer hypothesis} --- which states that programmers tend
to make simple mistakes, and the \emph{coupling effect} --- which states that test
cases capable of detecting faults in isolation continue to be effective even
when faults appear in combination with other faults~\cite{demillo1978hints}.
The coupling effect was first validated by Lipton et
al.~\cite{lipton1978the,demillo1978hints} up to fourth order by sampling.
Offutt~\cite{offutt1989thecoupling,offutt1992investigations} investigated second
order exhaustively, and sampled until $3^{rd}$ order mutants.
Langdon et al.~\cite{langdon2010efficient} using 85 first order mutants of
\emph{triangle} program and $16,383$ tests observed that tests that kill more
first order mutants kill proportionately more second and third order mutants.
Morell~\cite{morell1990atheory} provided a theoretical treatment
of fault based testing. Morell suggested that the competent programmer
hypothesis is same as \emph{alternate sufficiency} in fault based testing
which suggests that at least one of the alternates in the arena being considered
is the correct version, and also provided a formal treatment of the
coupling effect~\cite{morell1987amodel}
(i.e. the assumption is that fault masking has low probability of occurrence).
Morell~\cite{morell1983a} also provides a proof that no general algorithm exists
to identify existence of coupling between faults.
Wah et al.~\cite{wah1995fault,wah2000atheoretical,wah2001theoretical,wah2003ananalysis} using a
simple model of finite functions (the \emph{q-function} model, where \emph{q}
represents the number of functions thus composed) showed that the survival ratio of first and
second order test sets are respectively $\frac{1}{n}$ and $\frac{1}{(n^2 - n)}$
where $n$ is the order of the \finput~\cite{jia2011analysis}.
A major finding of Wah is that \emph{the coupling effect weakens as the system
size (in terms of number of functions in an execution path) increases (i.e. $q$ increases),
and it becomes unreliable when the system size nears the domain of functions}.
Another important finding was that \emph{minimization of test sets has a
detrimental effect}. That is, for $n$ faults, one should use $n$ test cases, with
each test case able to detect $n-1$ faults (rather than a single fault) to
ensure that the test suite minimizes the risk of missing higher order faults due
to fault masking.
Kapoor~\cite{kapoor2006formal} proved the existence of the coupling effect on
logical faults.
The competent programmer hypothesis was quantified recently by Gopinath et al.~\cite{gopinath2014mutations}.
Voas et al.~\cite{voas1993semantic} and later Woodward et al.~\cite{woodward2000testability}
suggests that the \emph{domain} to
\emph{range} ratio has an impact in hiding faults. Specifically, functions
with a high \emph{DRR} tend to mask faults. A similar measure is Dynamic Range to Domain
ratio studied by Al-Khanjari et al.~\cite{alkhanjari2003investigating}. They
found that for some programs there is a strong relationship between the
ratio and testability, but it is weak for others.

Androutsopoulos et al.~\cite{androutsopoulos2014an} suggested an
approach using information theoretic measures to study failed error
propagation. It was also found that one in ten tests suffered from failed
error propagation. Clark et al.~\cite{clark2012squeeziness} found that
likelihood of collisions were strongly correlated with an information theoretic
measure called \emph{squeeziness}, which is related to the amount of information
destroyed after applying the function to its input. Hence choosing a path for
tests that minimize squeeziness would reduce fault masking.

A fruitful area of research has been reducing the cost of mutation
analysis, broadly categorized as \textit{do smarter}, \textit{do
faster}, and \textit{do fewer} by Offutt et al.~\cite{offutt2001mutation}.
The \textit{do smarter} approaches include space-time trade-offs, weak
mutation analysis, and parallelization of mutation analysis. The \textit{do
faster} approaches include mutant schema generation, code patching, and
other methods to make the mutation analysis faster as a whole. Finally, the
\textit{do fewer} approaches try to reduce the number of mutants examined,
and include selective mutation and mutant sampling.

\begin{comment}
Various studies have tried to tackle the problem of approximating the full
mutation score without running a full mutation analysis.  The idea of
using only a subset of mutants (\textit{do fewer}) was conceived first
by Budd~\cite{budd1980mutation} and Acree~\cite{acree1980mutation}
who showed that using just 10\% of the mutants was sufficient to
achieve 99\% accuracy of prediction for the final mutation score. This
idea was further investigated by Mathur~\cite{mathur1991performance},
Wong et al.~\cite{wong1993mutation,wong1995reducing}, and Offutt et
al.~\cite{offutt1993experimental} using the Mothra~\cite{demillo1988an} mutation
operators for FORTRAN.  Lu Zhang et al.~\cite{zhang2010isoperator} compared
operator-based mutant selection techniques to random mutant sampling,
and found that random sampling performs as well as the operator selection
methods.  Lingming Zhang et al.~\cite{zhang2013operator} compared various
forms of sampling such as stratified random sampling based on operator strata,
stratified random sampling based on program element strata, and a combination
of the two. They found that stratified random sampling when strata were
used in conjunction performed best in predicting the final mutation score,
and as few as 5\% mutants was sufficient sample for a 99\% correlation with the actual
mutation score. Recently, it was found~\cite{gopinath2015howhard} that $9,604$
mutants were sufficient for obtaining $1\%$ accuracy for $99\%$ of the projects,
irrespective of the independence of mutants, or their total population.
\end{comment}

An important area of research when considering the coupling effect is that of
higher order mutants~\cite{jia2008constructing,jia2009higher,nguyen2014problems}. The important
idea here is that of subsuming higher order mutants, which are mutants that are
harder to kill than their constituent mutants. These are mutants where there is
at least a partial masking of effect of the first mutant by the second mutant.
While incidence of such mutants are low (as our results show), they are
important for the simple reason that they represent the hard to find bugs that
tend to interact, and hence represent a different class of bugs.

Our research is an extension of the theoretical work of Wah~\cite{wah2000atheoretical}
and Offutt~\cite{offutt1989thecoupling,offutt1992investigations}. The
major theoretical difference from Wah~\cite{wah2000atheoretical} is that, given
a pair of faulty functions that compose, we try to find the probability that,
for given test data, the second function masks the error
produced by the first one.  On the other hand, Wah~\cite{wah2000atheoretical}
tries to show that the coupling effect exists considering the \emph{entire}
program composed of \emph{q} functions, each having a single fault (given by
\emph{q} in the \emph{q}-function model).
% TODO: multi-fault functions. -- change in domain.
Next, Wah~\cite{wah2000atheoretical} assumes semantic separability of all
complex faults. However, as we show, there exist a class of complex faults that
are not semantically separable. We make this restriction clear.
%Wah~\cite{wah2000atheoretical}
%only considers functions which have same \finput $\mathbb{N}$ as input and output.
% Next, we use \emph{combinatory-calculus}, a formalism which is equivalent
% in power to any other representation of programs. Its use allows conversion from,
% and to, any other representation of computable programs. The use of
% \emph{combinatory-calculus} is important in that our analysis
% (and the analysis of Wah~\cite{wah2000atheoretical}) depends on separability of
% faults. This can be achieved \emph{if and only if} the faults can be separated
% in the combinatory calculus form. This restriction is not made precise in
% Wah~\cite{wah2000atheoretical}. The use of \lcalc and
% \emph{combinatory logic} also justifies the use of single parameter functions,
% which is assumed as given in Wah~\cite{wah2000atheoretical}.
Further, our
analysis shows that the probability of coupling is related to the \foutput,
not the \finput, as Wah~\cite{wah2000atheoretical} suggests. In fact,
Wah~\cite{wah2000atheoretical} considers only functions which have exactly same
\emph{domain} and \emph{range}, and hence are more restricted than our analysis.
Finally, we show that even if syntax is considered, our analysis results remains
valid.

With regard to the work by Offutt~\cite{offutt1992investigations}, the primary
difference is in the empirical relation we attempt to demonstrate. While
Offutt~\cite{offutt1992investigations} evaluates the traditional coupling
effect, and shows the empirical relation with respect to \emph{all} simple
faults and their combinations, we aim to demonstrate the \efaultT
and evaluate the relation between any pair of faults, and the combined fault
including both.



\section{Theory of Fault Coupling}
\label{sec:theory}
%The theory of coupling was investigated by Wah~\cite{wah2000atheoretical}, who
%proposed a simplified \emph{q}-function model (consisting of \emph{q} composed
%functions) for theoretical analysis.

We start with a functional view of programs where programs are constructed by
composition of different functions (this an assumption that was also adopted by Wah~\cite{wah2003ananalysis}).
While Wah considered composition of \emph{q} functions, which may have as many
as \emph{q} faults, we consider only pairs of faulty functions. This is because
any faulty program with a number of separable faults can be modeled as
composition of two faulty functions, which may themselves contain complex
separable faults. Hence, a theoretical evaluation of faulty function pairs is
sufficient to model higher order faults.

%\greybox{
We note that our analysis
is subject to these major assumptions,
also made by Wah~\cite{wah2000atheoretical}:
in the initial phase, we assume that we can model programs as mathematical functions. This
is the biggest assumption that we make, since there can be an infinite number
of syntactical equivalent alternatives to any given program.
% To make it tractable, we assume that syntax has a uniform impact on faulty and non-faulty
% versions, and modeling programs as functions reasonable.
We assume finite \finput and \foutput for functions considered, and only total
functions. We assume that faulty versions of a function have same \finput and
\foutput as that of the non-faulty version\footnote{
In practical terms, the (\finput,\foutput) of a function is its type signature.
It is possible that the \emph{range} (or \emph{image}) of a faulty version may
differ from the non-faulty version. However, we assume that a valid alternate
that survives the compiler will
have same type signature, and hence same \foutput.
}.
% Next, we assume a single faulty input out of the entire \emph{domain}. This need
% not be true in the real world. -- TODO: do we need this assumption?
Further, we model single parameter functions. We note that any function can be
regarded as a single parameter function by considering their input as composed
of a tuple of all the original parameters. That is, we consider a function with
two inputs -- say $a$ and $b$ as a function taking a tuple: $(a,b)$ as input,
with size of \emph{domain} given by $dom(a) \times dom(b)$.

A significant difference from the standard \qfunction model of Wah is that,
while Wah considers how a known number of test inputs (1, 2, 3 and
higher) some of which can detect some of the component faulty functions (there
can be as many as $q$ faulty functions) can together detect the composite faulty
function, we consider the probability of any single test input that can detect a
fault being masked by the addition of a new fault.
% Further,
% Wah derives the expected number of survivors, while we determine the average
% survival ratio of a fault.
This allow us to significantly simplify our analysis.

% Finally we assume the separability of fault pairs.
In order for our theory to be applied, the fault pair needs to be separated out
such that first function and second function can be clearly demarcated. The
theory relies on the \foutput of first function being the \finput of the second.
This means that the influence of first should be limited to the input parameter
for the second, and the first completely independent of the second (we show
later how this condition may be relaxed). There are various places where this
assumption may not hold. Note that the theory \emph{does not} rely on the
constituent faults being considered to be simple.

A major idea in our analysis is the semantic separability of faults. Two faults
present in a function are said to be separable \emph{if and only if} the
smallest possible chunk containing both faults can be decomposed into two
functions $g$ and $h$ such that each fault is isolated within a single function
(providing $g_a$ and $h_b$ as faulty functions), the behavior of composition
$h \circ g$ equals the behavior of the original chunk in terms of input and
output , and composition $h_b \circ g_a$ equals the behavior of the function
with both faults. A \emph{chunk} here is any small section of the program that can
be replaced by an independent function preserving the behavior.

That is, given a function:
\begin{lstlisting}
def functionX(x, y, n)
  for i in (1..n):
      y = faultyA(x)              (1)
      if odd(i): x = faultyB(y)   (2)
      x += 1
\end{lstlisting}
The lines (1) and (2) together form a chunk.
The interaction between the
faults and their separability is discussed next.
%}

\subsection{Interaction Between Faults}
We define two kinds of interaction between faults here: \emph{weak}
interaction, and \emph{strong} interaction. \emph{Weak}
interactions occur when faults can be isolated into different
functions such that the output of the function containing the first fault is
the input of a function containing the second fault.

That is, given two faults $\hat{a}$ and $\hat{b}$ in a function $f$, which
can be split into $f_{ab} =  h_{b} \circ g_{a}$, where $g_{a}$ and $h_{b}$
are faulty functions, the only interaction between $\hat{a}$ and $\hat{b}$ is
because the fault $\hat{a}$ modifies the input of $h$ (or $h_{b}$) from $g(i_0)$
to $g_{a}(i_0)$ (where $i_0$ is an input for $f$). That is, the interaction
can be represented by a modified input value.

\emph{Strong} interactions happen when the interpretation of the second
fault is affected by the first fault. In this case, we cannot separate them
out into independent faulty functions.
% For example, consider the \lcalc expression $\lambda\,x\,y\,.\,x\,y$
% and a corresponding faulty expression $\lambda\,y\,x\,.\,x\,y$.
% Here, while there are two faults ($x \rightarrow y, y \rightarrow x$), they
% cannot be separated into isolated composing functions because the interpretation of the
% second is dependent on whether the first fault is present or not.
For a practical example, consider the following function in Python:

\begin{lstlisting}
  def swap(x,y): x,y=y,x
\end{lstlisting}
Say this was mutated into
\begin{lstlisting}
  def swap(x,y): x,y=x,y
\end{lstlisting}
Clearly, there were two independent lexical changes: $x \rightarrow y$
and $y \rightarrow x$. However, the
disassembly of the original program is given by
\begin{lstlisting}
>>> dis.dis(swap)
  1   0 LOAD_FAST       1 (y)
      3 LOAD_FAST       0 (x)
      6 ROT_TWO
      7 STORE_FAST      0 (x)
     10 STORE_FAST      1 (y)
     13 LOAD_CONST      0 (None)
     16 RETURN_VALUE
\end{lstlisting}
It can be seen that the changes in Python resulted in intertwined changes,
and hence cannot be separated out.
Since the faults cannot be separated out, \emph{strong} interactions produce
faults that have a different characteristic from the simple faults from which
they are built, and hence should be considered as independent atomic
faults\footnote{We note that Wah's analysis~\cite{wah2000atheoretical} assumes
that all complex faults can be separated, and does not take into consideration
such strongly interacting faults.}.
Why should we consider the semantically inseparable faults as independent
faults? An intuitive argument is to consider two functions that
implement $id$ (these are not strongly interacting).
That is, given any value $x$, we have $g(x) = h(x) = x$. If two faults $\hat{a}$, and
$\hat{b}$ occur as we suggest above in $g$ and $h$, causing inputs $i$ to $g_a$
and inputs $j$ to $h_b$ to fail, then the faulty inputs for $h_b \circ g_a$ are
bounded by $i \cup j$, where $i$ represents inputs to $f$ that result in faulty
outputs due to faults in $g$ and $j$ represents inputs to $f$ resulting in faulty
outputs due to faults in $h$.

What about fault masking? Any input $i$ that failed for $g_a$ could possibly
result in an input value that would cause a failure for $h_b$. For any element
outside of $i$, there is no possibility of two faults acting on it, and hence
no possibility of fault masking.
However, if the faults are not semantically separable, one cannot make these
guarantees, as the faulty inputs may be larger than $i \cup j$ or even
completely different. In the general case, when the interaction is weak, we
expect the faulty output for up to $i \cup j$.
These are represented in Figure~\ref{fig:rep}.

This is simple enough to prove formally. Consider a function $f$ that has \finput $x$. Say
that it can be represented as $h \circ g$ using two independent functions.
Replacing $g$ with $g_a$ causes $i \in x$ inputs to result in faults.
Similarly, replacing $h$ with $h_b$ causes $j \in x$ inputs to $f$ to result in
faults. Joining together to form $f_{ab}$, we know that any of $i \in x$ has a
potential to produce a faulty output unless it was masked by $h_b$. Similarly,
any of $j \in x$ also has the possibility of producing a faulty output.
Now, consider any element $k$ not in either $i$ or $j$. It will not result in
a faulty output while passing through $g_a$ because it is not in $i$, further,
the value $g_a(k) = g(k) = k_1$. We already know that $k_1$ would not result in
a faulty output from $h_b$ because $k \notin j$. Hence, any element
$k \notin i \cup j$ will not be affected by faults $\hat{a}$ and $\hat{b}$.

Remember that we can make this assertion only because we can replace $g$ and
$h$ separately. Suppose it is a strong interaction between $\hat{a}$ and $\hat{b}$.
In this case, any function could potentially be a replacement for $f$. Hence, any
element in $x$ may potentially result in a fault when $f_{ab}$ is applied. Indeed,
we are not the first ones to make this observation. These kinds of interactions
where the higher order faults have a different behavior (and result in different
inputs with faulty outputs) than the component
faults are called de-coupled higher order mutants by Harman et al.~\cite{harman2010manifesto}.

%---
\begin{comment}
For example consider the following function, and the test cases:
\begin{lstlisting}
  def not3(a):
 o)  return a != 3
m1)  return a >  3
m2)  return a != 2
m3)  return a >  2

                    m1 m2 m3
                    --------
t1) not3(1) = T :   F, T, T

t2) not3(4) = T :   T, F, T
t3) not3(5) = T :   T, T, T
\end{lstlisting}

As before, $m1$ and $m2$ are constituent faults of $m3$. The test $t1$ detects
$m1$, and test $t2$ detects $m2$. But $m3$ is triggered by neither alone, but
combined to $m3$ it triggers $t3$.

\end{comment}
%---
Depending on the syntax and semantics of the language used, there may be
different language features
(another common example is the order of arguments to a function)
that cause faults to be strongly interacting.

\subsection{Analysis}
%This suggests that,
Say we have a program $f$, where we
have two simple faults $\hat{a}$, and $\hat{b}$, which can be
applied to $f$ to produce two functions $f_a$ and $f_b$ containing one fault
each, and $f_{ab}$ containing both faults (Figure~\ref{fig:lines}).

Say such a program can be partitioned into two functions $g$ and $h$ such that
\[
  f = h \circ g
\]
with added restriction that the fault $\hat{a}$ lies in function $g$,
producing an alternative function $g_a$, and the
fault $\hat{b}$ lies in function $h$ producing $h_b$, such that the new faulty
version of $f$ containing both is given by
\[
  f_{ab} = h_b \circ g_a
\]
We note that the particular kind of fault depends on the syntax and semantics
of the programming language used, and there can be fault pairs that cannot be
separated cleanly. As stated previously, we ignore these kinds of fault pairs
as they are syntax dependent and strongly interacting. Hence, no general
solution is possible for these faults.

The basic question we are trying to resolve is, given that we can distinguish
a fault in isolation using a given input, what is the probability that
another fault would not result in the masking of that fault for the same input?

That is, we are given a test input $i_0$ for $f$, that can distinguish between ($f$, $f_a$).
%, and ($f$, $f_b$).
The question is, what is the probability that ($f$, $f_{ab}$) can
be distinguished by the same input?

Since we know that $f_a$ is
% , and $f_b$ are
distinguished from $f$, we know that $g_a(i_0) \neq g(i_0)$. Hence, the function
$h_b$ will have a different input than $h$. Thus, the question simplifies to:
given an alternate input for function $h$ (or anything that can be substituted
in its place), what is the probability that a faulty $h$, with the new input
$g_a(i_0)$ will result in same output as the old $h$, with the old input
$g(i_0)$?

Let us assume for simplicity\footnote{
There are two factors that make this assumption reasonable: first,
for real world functions, there is almost always a limit on the \finput and
\foutput whether it be due to the data type used, or due to constraints such
as the available resources such as memory. Second, as we show later, the size
of the \foutput is actually inversely proportional to the masking effect,
which helps our case as the \foutput increases (note that
\finput will always be larger than \foutput). We also note that the assumption of
finite \finput was also made by Wah~\cite{wah2000atheoretical}.
} that functions $g$ and $h$ have fixed \finput
and a \foutput given by
\begin{align*}
g \in G &: \mathbb{L} \rightarrow \mathbb{M} \\
h \in H &: \mathbb{M} \rightarrow \mathbb{N} \\
\end{align*}
That is, $h$ belongs to a set of functions $H$, which has a \finput $M$, and a
\foutput $N$ such that $m = |M|$ and $n = |N|$. Considering all possible functions
in $H$, with the given \finput and \foutput, there will be $n^m$ unique functions
in $H$ (separated by at least one different $\{input,output\}$ pair).

The only constraint on $h_b$ we have is that $h_b(g_a(i_0))$ should result in
the exact same output as $h(g(i_0))$. That is, for some particular input for
$h_b$, the output is fixed at a particular output. We are looking for
functions that can vary in every other $\{input,output\}$ pair except for the
pair given by $\{g_a(i_0), h(g(i_0)\})$. There are $n^{m-1}$ functions that
can do that out of $|H| = n^m$ functions. That is, the \kappaT is
given by
\[
  \kappa = 1 - \frac{n^{m-1}}{n^m}
\]
This is simplified to $ 1 - \frac{1}{n}$ of the total number of eligible functions
where $m$ is the size of \finput, and $n$ is the size of \foutput of the function. 
That is, given any test input, the probability of the \couplingC effect where
the fault in one constituent is not masked by the fault in another is $1 - \frac{1}{n}$,
and $\frac{1}{n}$ tends to be very small when the \foutput of the function ($n$) is
large.

A symmetric argument can be made when the function fixed is $h$, and $g$
varies. There are $m^l$ functions in $G$, of which $m^{l-1}$ can be used as
a replacement without affecting $\{input, output\}$, in which case, the
probability of composite coupling effect is $1 - \frac{1}{m}$
where $m$ is the \foutput
\footnote{We note that the logic of probability is very similar to
Wah~\cite{wah2000atheoretical}, and this is the same value derived by Wah
for single test input, where $n$ is the \finput of the function as Wah
does not consider functions that have a different \finput and \foutput.}.

For example, in the case of a function with its \finput and \foutput restricted to
boolean, with \finput $2$ and \foutput $2$, the probability of \couplingC effect
is $1 - \frac{1}{2} = 0.5$, and in the
case of an octet ($8$ bits), the probability of \couplingC effect
is $0.9961$ (given equal probability for all faults).

This is easy to verify in the case of booleans: There are $2^2$ possible unique
boolean functions. $f_{id}$ is just the identity function, and $f_{not}$ is the $not$
function. $f_{true}$ always returns $true$, and $f_{false}$ always returns
$false$. Consider a combination of functions such as $f = f_{id} \circ f_{not}$.
We consider a test input $true$, which will be converted to $false$ by
function $f$. Say there is an error in $f_{id}$, and we mistakenly used
$f_{false}$ instead (note that we would not be able to use $f_{true}$ since we
assume that the original input would have detected the error if it was taken
individually). The question is, how many ways can you make a faulty
$f_{not}$ such that we get $false$? That is, which functions map $false$ to $false$?
There are just two -- $f_{id}$, and $f_{false}$ which is 50\% of the available
functions.


\begin{figure}
\begin{tikzpicture}
\node[text width=3cm] at (1,1) {$h \circ g$};
\node[text width=3cm] at (0.5,-1) {$g$};
\node[text width=3cm] at (0.5,-3) {$h$};

\draw (0,0) circle [x radius=1.4cm, y radius=6mm];
\draw (0,-2) circle [x radius=1.2cm, y radius=5mm];
\draw (0,-4) circle [x radius=1cm, y radius=4mm];

\draw[-,blue] (0,0) circle (1pt) -- (0,-2) circle (1pt) node[pos=.5,above] {} node[pos=-.1]{a};
\draw[-,blue] (0,-2) circle (1pt) -- (-0,-4) circle (1pt) node[pos=1.5,below] {} node[pos=-1.1,yshift=-6mm,near end]{};

\draw[-,blue] (-.3,0)  circle (1pt) -- (-0.3,-2) circle (1pt) node[pos=.5,sloped,above] {} node[pos=-.1]{b};
\draw[-,blue] (-0.3,-2)  circle (1pt) -- (-0.2,-4) circle (1pt) node[pos=.5,sloped,above] {} node[pos=-1.1,yshift=-6mm,near end]{};

\draw[-,blue] (-.6,0)  circle (1pt) -- (-0.5,-2) circle (1pt) node[pos=.5,sloped,above] {} node[pos=-.1]{c};
\draw[-,blue] (-0.5,-2)  circle (1pt) -- (-0.4,-4) circle (1pt) node[pos=.5,sloped,above] {} node[pos=-1.1,yshift=-6mm,near end]{};

\draw[-,blue] (-.9,0)  circle (1pt) -- (-0.7,-2) circle (1pt) node[pos=.5,sloped,above] {} node[pos=-.1]{d};
\draw[-,blue] (-0.7,-2)  circle (1pt) -- (-0.6,-4) circle (1pt) node[pos=.5,sloped,above] {} node[pos=-1.1,yshift=-6mm,near end]{};

\draw[-,blue] (.3,0)  circle (1pt) -- (0.2,-2) circle (1pt) node[pos=.5,sloped,above] {} node[pos=-.1]{e};
\draw[-,blue] (0.2,-2)  circle (1pt) -- (0.2,-4) circle (1pt) node[pos=.5,sloped,above] {} node[pos=-1.1,yshift=-6mm,near end]{};

\draw[-,blue] (.6,0)  circle (1pt) -- (0.4,-2) circle (1pt) node[pos=.5,sloped,above] {} node[pos=-.1]{f};
\draw[-,blue] (0.4,-2)  circle (1pt) -- (0.4,-4) circle (1pt) node[pos=.5,sloped,above] {} node[pos=-1.1,yshift=-6mm,near end]{};

\draw[-,blue] (.9,0)  circle (1pt) -- (0.6,-2) circle (1pt) node[pos=.5,sloped,above] {} node[pos=-.1]{g};
\draw[-,blue] (0.6,-2)  circle (1pt) -- (0.6,-4) circle (1pt) node[pos=.5,sloped,above] {} node[pos=-1.1,yshift=-6mm,near end]{};

\node[text width=3cm] at (2 + 1,-1) {$g'$};
\node[text width=3cm] at (2 + 1,-3) {$h'$};
%---------------------
\draw (3 + 0,0) circle [x radius=1.4cm, y radius=6mm];
\draw (3 + 0,-1.5) circle [x radius=1.2cm, y radius=3mm];
\draw (3 + 0,-2.5) circle [x radius=1.2cm, y radius=3mm];
\draw (3 + 0,-4) circle [x radius=1cm, y radius=4mm];

\draw[-,blue] (3 + 0,0) circle (1pt) -- (3 + 0,-2 + 0.5) circle (1pt) node[pos=.5,above] {} node[pos=-.1]{a};
\draw[-,red] (3 + 0,-2-0.5) circle (1pt) -- (3 + 0.2,-4) circle (1pt) node[pos=1.5,below] {} node[pos=-1.1,yshift=-6mm,near end]{};

\draw[-,blue] (3 + -.3,0)  circle (1pt) -- (3 + -0.3,-2 + 0.5) circle (1pt) node[pos=.5,sloped,above] {} node[pos=-.1]{b};
\draw[-,blue] (3 + -0.3,-2-0.5)  circle (1pt) -- (3 + -0.2,-4) circle (1pt) node[pos=.5,sloped,above] {} node[pos=-1.1,yshift=-6mm,near end]{};

\draw[-,blue] (3 + -.6,0)  circle (1pt) -- (3 + -0.5,-2 + 0.5) circle (1pt) node[pos=.5,sloped,above] {} node[pos=-.1]{c};
\draw[-,blue] (3 + -0.5,-2-0.5)  circle (1pt) -- (3 + -0.4,-4) circle (1pt) node[pos=.5,sloped,above] {} node[pos=-1.1,yshift=-6mm,near end]{};

\draw[-,red] (3 + -.9,0)  circle (1pt) -- (3 + -0.3,-2 + 0.5) circle (1pt) node[pos=.5,sloped,above] {} node[pos=-.1]{d};
\draw[-,blue] (3 + -0.7,-2-0.5)  circle (1pt) -- (3 + -0.6,-4) circle (1pt) node[pos=.5,sloped,above] {} node[pos=-1.1,yshift=-6mm,near end]{};

\draw[-,red] (3 + .3,0)  circle (1pt) -- (3 + 0.6,-2 + 0.5) circle (1pt) node[pos=.5,sloped,above] {} node[pos=-.1]{e};
\draw[-,blue] (3 + 0.2,-2-0.5)  circle (1pt) -- (3 + 0.2,-4) circle (1pt) node[pos=.5,sloped,above] {} node[pos=-1.1,yshift=-6mm,near end]{};

\draw[-,blue] (3 + .6,0)  circle (1pt) -- (3 + 0.4,-2 + 0.5) circle (1pt) node[pos=.5,sloped,above] {} node[pos=-.1]{f};
\draw[-,blue] (3 + 0.4,-2-0.5)  circle (1pt) -- (3 + 0.4,-4) circle (1pt) node[pos=.5,sloped,above] {} node[pos=-1.1,yshift=-6mm,near end]{};

\draw[-,blue] (3 + .9,0)  circle (1pt) -- (3 + 0.6,-2 + 0.5) circle (1pt) node[pos=.5,sloped,above] {} node[pos=-.1]{g};
\draw[-,red] (3 + 0.6,-2-0.5)  circle (1pt) -- (3 + 0.4,-4) circle (1pt) node[pos=.5,sloped,above] {} node[pos=-1.1,yshift=-6mm,near end]{};

%---------------------
\node[text width=3cm] at (7,1) {$h' \circ g'$};
\draw (6 + 0,0) circle [x radius=1.4cm, y radius=6mm];
\draw (6 + 0,-2) circle [x radius=1.2cm, y radius=3mm];
\draw (6 + 0,-4) circle [x radius=1cm, y radius=4mm];

\draw[-,red,dashed] (6 + 0,0) circle (1pt) -- (6 + 0,-2 ) circle (1pt) node[pos=.5,above] {} node[pos=-.1]{a};
\draw[-,red] (6 + 0,-2) circle (1pt) -- (6 + 0.2,-4) circle (1pt) node[pos=1.5,below] {} node[pos=-1.1,yshift=-6mm,near end]{};

\draw[-,blue] (6 + -.3,0)  circle (1pt) -- (6 + -0.3,-2 ) circle (1pt) node[pos=.5,sloped,above] {} node[pos=-.1]{b};
\draw[-,blue] (6 + -0.3,-2)  circle (1pt) -- (6 + -0.2,-4) circle (1pt) node[pos=.5,sloped,above] {} node[pos=-1.1,yshift=-6mm,near end]{};

\draw[-,blue] (6 + -.6,0)  circle (1pt) -- (6 + -0.5,-2 ) circle (1pt) node[pos=.5,sloped,above] {} node[pos=-.1]{c};
\draw[-,blue] (6 + -0.5,-2)  circle (1pt) -- (6 + -0.4,-4) circle (1pt) node[pos=.5,sloped,above] {} node[pos=-1.1,yshift=-6mm,near end]{};

\draw[-,red] (6 + -.9,0)  circle (1pt) -- (6 + -0.3,-2 ) circle (1pt) node[pos=.5,sloped,above] {} node[pos=-.1]{d};
\draw[-,blue,dotted] (6 + -0.7,-2)  circle (1pt) -- (6 + -0.6,-4) circle (1pt) node[pos=.5,sloped,above] {} node[pos=-1.1,yshift=-6mm,near end]{};

\draw[-,red,dotted] (6 + .3,0)  circle (1pt) -- (6 + 0.6,-2 ) circle (1pt) node[pos=.5,sloped,above] {} node[pos=-.1]{e};
\draw[-,blue,dotted] (6 + 0.2,-2)  circle (1pt) -- (6 + 0.2,-4) circle (1pt) node[pos=.5,sloped,above] {} node[pos=-1.1,yshift=-6mm,near end]{};

\draw[-,blue] (6 + .6,0)  circle (1pt) -- (6 + 0.4,-2 ) circle (1pt) node[pos=.5,sloped,above] {} node[pos=-.1]{f};
\draw[-,blue] (6 + 0.4,-2)  circle (1pt) -- (6 + 0.4,-4) circle (1pt) node[pos=.5,sloped,above] {} node[pos=-1.1,yshift=-6mm,near end]{};

\draw[-,red,dashed] (6 + .9,0)  circle (1pt) -- (6 + 0.6,-2 ) circle (1pt) node[pos=.5,sloped,above] {} node[pos=-.1]{g};
\draw[-,red] (6 + 0.6,-2)  circle (1pt) -- (6 + 0.4,-4) circle (1pt) node[pos=.5,sloped,above] {} node[pos=-1.1,yshift=-6mm,near end]{};


\end{tikzpicture}
\caption{Weak fault interaction: The first function \emph{g} is faulty at inputs \emph{d} and \emph{e}, while second function \emph{h} is faulty at inputs \emph{a} and \emph{g}. Hence, we expect \emph{d, e, a, g} to produce faulty outputs for the combined function \emph{$h' \circ g'$} if only weak interaction is considered. The input \emph{e} has a possibility of being masked by the fault at \emph{g} in the second function \emph{h}.}
\label{fig:rep}
\end{figure}


\subsection{Recursion and Iteration}

\begin{figure}
\centering
\begin{tikzpicture}
\draw[-|,blue] (0,-1.5) -- (3.5,-1.5) node[pos=.5,sloped,above] {$\frac{1}{n}$} node[pos=-.1,sloped]{$i_0$};
\draw[-|,red,dotted] (0,-1.5) circle (2pt) -- (3.5,-3) node[pos=.5,sloped,below] {$\frac{n-1}{n}$};
\draw[|-|,blue] (3.5,-1.5) -- (7,-1.5) node[pos=.2,sloped,above] {$\frac{1}{n}\times \frac{1}{n}$} node[pos=1.2]{$\frac{1}{n}$};
\draw[|-|,red,dotted] (3.5,-1.5) -- (7,-3) node[pos=.4,sloped,above] {$\frac{n-1}{n}\times \frac{1}{n}$} node[pos=1.2]{};

\draw[|-|,red,dotted] (3.5,-3) -- (7,-3) node[pos=.5,sloped,above] {$\frac{n-1}{n}\times\frac{n-1}{n}$} node[pos=1.2]{$\frac{n-1}{n}$};
\draw[|-|,blue] (3.5,-3) -- (7,-1.5) node[pos=.2,sloped,above] {$\frac{1}{n} \times \frac{n-1}{n}$} node[pos=1.2]{};
\end{tikzpicture}
\caption{Recursive interaction. The \emph{blue solid} lines represent the masking values where the values are same as what would be expected before the fault was introduced, and the \emph{red dotted} lines represent values that are different from the non-faulty version so that faults could be detected.}
\label{fig:recursion}
\end{figure}

A common feature of general purpose languages is \emph{recursion} and
\emph{iteration}. These present challenges to our analysis. For example,
consider the loop below.

\begin{lstlisting}
  while y > 0: y = h(g(y)
\end{lstlisting}

Here, the two functions $g$ and $h$ are otherwise independent. However, the
input of $h$ influences $g$, and vice versa. Here, we do not know when the
loop will end, and any faults will be detected. The faults may be detected
after a larger or smaller number of iterations than the non faulty version
of the program. Hence, what we can do, is to consider the chances of
propagation of the faulty value after each iteration. That is, if a faulty
value is present after executing the function $g_a$ once, what are the
chances that it will be caught at the end of each iteration?

Let us use $f$ to denote the program segment composed of $g$ and $h$, as before.
After the first iteration of $f$, we will have $\frac{1}{n}$ possibility
for fault masking as we discussed before, and $\frac{n-1}{n}$ possibility
for detectable faulty values. Now, consider the next iteration. In this case,
of the original $\frac{1}{n}$ masked outputs, $\frac{1}{n}$ will again be
masked, for a total of $\frac{1}{n^2}$, and the remaining $\frac{(n-1)}{n^2}$
will have a value that is faulty. Consider the original $\frac{n-1}{n}$ that
had faulty values in the first iteration. Out of that, $\frac{1}{n}$ will
be masked in the second iteration (i.e. $\frac{n-1}{n^2}$). Similarly,
$\frac{(n-1)^2}{n^2}$ of the original faulty outputs will remain faulty.
That is, after second iteration, we will have $\frac{1}{n^2} + \frac{n-1}{n^2} = \frac{1}{n}$
masked output values. Similarly, we will have $\frac{n-1}{n^2} + \frac{(n-1)^2}{n^2} = \frac{1-n}{n}$
faulty output values. That is, after each iteration, we will have $\frac{1}{n}$
possibility of fault masking (See Figure~\ref{fig:recursion}). Hence, \faultT will hold even for recursion and
iteration.

A question may be asked, what happens for functions where the iteration may
proceed in different paths during different executions. For example:

\begin{lstlisting}
  for i in 1..10:
    if odd(i): x = g(y)
    else: y = h(x)
\end{lstlisting}

In programs such as this, one may unroll the loop, i.e.
\begin{lstlisting}
  for i in 1..10:2:
      x = g(y)
      y = h(x)
\end{lstlisting}
which can make it amenable to the above treatment. Recursion can also be
treated in a same fashion. We do not claim that this is exhaustive. There could
exist other patterns of recursion or iteration that do not fit this template.
However, most common patterns of recursion and iteration could be captured in
this pattern.

Can we extend the bounds we found ($i \cup j$ for faulty outputs) to recursion?
Unfortunately, it is possible for a faulty function to interact with its own
output during recursion, and hence mask a failure. Hence, we can not bound the
failure causing inputs in a doubly faulty function that incorporates recursion.

\subsection{Accounting for Multiple Faults}
What happens when there are multiple faults? Say, we have a system modeled by
$p \circ q \circ r \circ s \circ t \circ u$, where any of the functions may be
faulty or not faulty, for example
$p_a \circ q \circ r_b \circ s_c \circ t_d \circ u$. We can not directly apply
the technique in recursion because there are non-faulty functions interspersed.
The thing to remember here is that a non faulty function immediately adjacent to
a faulty function can together be considered a faulty function. Hence, the above
reduces to $(p_a \circ q) \circ r_b \circ s_c \circ (t_d \circ u)$, or equivalently
$pq_a \circ r_b \circ s_c \circ tu_d$. This is now
amenable to the treatment in Figure~\ref{fig:recursion} because each function
now can produce $\frac{1}{n}$ non-faulty and $\frac{n-1}{n}$ faulty outputs. An
additional complication is that a general expression is not possible unless we
simplify further, and assumes that \finput and \foutput of all functions are same. 
With this simplification, even when we consider a number of faulty functions,
the mean ratio of fault masking remains the same at $\frac{1}{n}$.

Indeed, we note that this is one of the significant differences from Wah's
analysis. Wah does not attempt to collapse the non-faulty functions to their
neighbours. Why do we do this? Because we know that each faulty function on
their own was detected by the test suite. That is, we know that
$p \circ q_a \circ r \circ s \circ t \circ u$ would have been detected.
Hence, we can certainly consider $pq_a \circ r \circ s \circ t \circ u$ as
the set of functions where the function $pq_a$ is the faulty function with
an atomic fault.

\subsection{Dynamically Checked Languages}

How can we apply the composite coupling hypothesis to \emph{dynamically checked}
languages, or \emph{unityped} languages? In both cases, any single function can
be taken to have the exact same \finput and \finput, the size of which may be
taken to be extremely large (but finite due to constraints of the environment).
Intuitively, this is because any function may be replaced by any other, and
one may not identify a faulty input type until execution passes through that
function. Hence in such languages, we can expect the \kappaT to be large.

\subsection{Impact of Syntax}

\begin{figure}
\centering
\begin{tikzpicture}
\draw[-|,blue] (0,0) circle (2pt) -- (2,0) node[pos=.5,sloped,above]{$g$} node[pos=-.1,sloped]{$i_0$};
\draw[|-|,blue] (2,0) -- (4,0) node[pos=.5,sloped,above] {$h$} node[pos=1.3,sloped]{$f(i_0)$};

\draw[-|,red] (0,-1.5) circle (2pt) -- (2,-2) node[pos=.5,sloped,below] {$g_a$};
\draw[-|,blue,dotted] (0,-1.5) -- (2,-1.5) node[pos=.5,sloped,above] {$g$} node[pos=-.1,sloped]{$i_0$};
%\draw[|-|,red,dotted] (2,-1.5) -- (4,-1) node[pos=.5,sloped,above] {$g_a$} node[pos=1.3]{$f_a(i_0)$};

\draw[|-|,red] (2,-2) -- (4,-2.5) node[pos=.5,sloped,below] {$h_b$} node[pos=1.3]{$f_{ab}(i_0)$};
\draw[|-|,blue,dotted] (2,-2) -- (4,-2) node[pos=.7,sloped,above] {$h$} node[pos=1.3]{$f_b(i_0)$};
\draw[|-|,black,dotted] (2,-2) -- (4,-1.5) node[pos=.5,sloped,above] {$h_{b'}$} node[pos=1.65]{$f_{ab}(i_0) = f(i_0)$};
\end{tikzpicture}
\caption{Fault interaction ($g_a(i_0)$ is masked by $h_{b'}$)}
\label{fig:lines}
\end{figure}

In order to model \couplingC, we relied on a simplification ---
that all faults are equally probable. However, in the real world this is often not
the case, with faults that are closer syntactically being more probable than
faults which are not in the syntactic neighborhood of correctness.
In fact, we have some
reasonable estimate of the distribution of size of faults that programmers
make~\cite{gopinath2014mutations}.

Implementation of functions as code
need not necessarily follow the same distribution as that of their mathematical
counterparts. For example, for mathematical functions, there exist only 4
functions that map from a boolean to a boolean. However, there can be an
infinite number of program implementations of that function.
The way it can be made tractable is again to consider the human element.
The \emph{competent programmer hypothesis} suggests that faulty programs are
close (syntactically) to the correct versions. So one need only consider a
limited number of alternatives (the number of which is a function of the size
of the correct version, if one assumes that each token may be legally replaced
by another).
% One could extend it to suggest
% that a human programmer would also tend to prefer small programs that can
% implement a given functionality, and this assumption can help make the number
% of alternatives we consider tractable (the number of such alternatives then
% becomes a (possibly exponential) function of the smallest program size).

As soon as we speak about syntactic neighborhood,
the syntax of a language can have a large influence on which faults can be considered to be in a
neighborhood. However, we note that most languages seem to follow a similar
distribution of faults with a size below 10 tokens for 90\% of faults~\cite{gopinath2014mutations}.

Let us call the original input
to $h$, $g(i_0) = j_0$, and the changed value $g_a(i_0) = j_a$.
Similarly, let $f(i_0) = k_0$, $f_a(i_0) = k_a$, $f_b(i_0) = k_b$,
and $f_{ab}(i_0) = k_{ab}$. Given two inputs $i_0$, and $i_1$ for
a function $f$, we call $i_0$, and $i_1$ \emph{semantically close} if
their execution paths in $f$ follow equivalent profiles, e.g., taking the same
branches of conditionals. We call $i_0$ and $i_1$ \emph{semantically far}
in terms of $f$ if their execution profile is different.

Consider the possibility of masking the output of $h_b$ by $g_a$
($g_{a'}$ in Figure~\ref{fig:lines}). We already
know that $h(j_a) = k_b$ was detected. That is, we know that
$j_a$ was sufficiently different from $j_0$, that it propagated through $h(j_a)$ to be caught by a test case. Say $j_a$ was \emph{semantically far}
from $j_0$, and the difference in code path contained the fault $\hat{a}$.
In that case, the fault $\hat{a}$ would not have been executed, and
since $k_{ab} = k_b$, it will always be detected.

On the other hand, say $j_a$ was \emph{semantically close} to $j_0$ in terms of
$g$ and the fault $\hat{a}$ was executed. There are again three possibilities.
The first is that $\hat{a}$ had no impact, in which case the analysis
is the same as before. The second is that $\hat{a}$ caused a change in the output.
It is possible that execution of $\hat{a}$ could be problematic enough to always
cause an error, in which case we have $k_{ab} = k_a$, and detection.
Thus masking requires $k_{ab}$ to be equal to $k_0$.

Even if we assume that the function $h_b$ is close syntactically to $h$, and
that this implies semantic closeness of functions $h$ and $h_b$, we expect
the value $k_{ab}$ to be near $k_b$, not $k_0$.
This suggests that masking, even when considered in the light of
syntactical neighborhood, is still unlikely, but this belief requires
empirical verification since we are unable to assign probabilities to
the cases above.  Our empirical data (provided in the next section of
this paper) should shed light on the actual incidence of masking when
syntactic/semantic neighborhoods are taken into account, since real
faults are likely in the syntactic and semantic neighborhood of the
correct code.

A statistical observation can further buttress our argument. We know that
if all functions were equally probable, fault masking has low probability.
Now, consider the functions that are syntactically close to a given function.
For most input values, we can assume that the syntactically close functions
will have same output as that of the given function, more so than functions
that are far away lexically. If $h$ did not mask a value originally, (which
we know since we were able to detect fault $h(g_a(i_0))$), then the
syntactically close functions to $h$ will with a higher probability than a
uniform sample, produce the same value as $h(g_a(i_0))$, which will be detected
as faulty.


\subsection{Can Strong Interaction be Avoided?}

Remember that the argument of mutation analysis is that if a test suite can find
all atomic faults, then by \faultT, a large percentage ($\kappa$) of
complex faults will also be found by the same test suite. A problem with this
is that one can in general never be sure that all atomic faults have been found,
as any fault interaction has the potential to produce an atomic fault through
strong interaction.
Hence, it is interesting to ask, what features in a programming language
contribute to strong fault interaction, and is there a representation of
programs where strong fault interaction is guaranteed to be absent.

The first question is, given that the strong interaction is dependent on the
execution, can runtime environment or compiler order computation so that strong
interaction is no longer present?

Consider the function \emph{swap (a,b) = (b,a)} that we examined earlier. We
see how one may mistakenly use \emph{id (a,b) = (a,b)} instead, and cause a
strong interaction. Now, the question is, does there exist a way to split the
two functions, so that the condition of separability can be satisfied?
Given that there are only four possible functions that can operate on a tuple,
(\emph{swap (a,b) = (b,a)}, \emph{id (a,b) = (a,b)}, \emph{dupleft (a,b) = (a,a)}, \emph{dupright (a,b) = (b,b)})
we could check it exhaustively. The condition is that the functions representing
single faults should individually cause a detectable deviation on their own, and
on composition, result in same behavior as \emph{id}. Now, it can be seen that,
neither of the single fault functions can behave like $swap$ since that
represents \emph{no} fault, so they can not behave like $id$, since that suggests that
the other faulty function behaves like $swap$. Hence, no compiler or runtime
environment can remove the strong interaction in \emph{swap}.

The next question is, where can we expect strong interaction to appear?
While we can not provide an exhaustive overview of possible language features,
we can demonstrate that even very simple languages such as the \lcalc are vulnerable
to strong interaction. For example, consider the \lcalc expression $\lambda x\,y\,. y\,x$,
and its faulty version $\lambda x\,y\,. x\,y$. There are two lexical points where
the faults have been injected $\{x \rightarrow y, y \rightarrow x\} $. However,
they cannot be separated out. That is, even simple features such as variables
can cause strong interaction.



\section{Methodology for Assessment}
\label{sec:methodology}

Our methodology was guided by two principles~\cite{siegmund2015views}:
We sought to minimize the number of variables that were in play, striving to
remove the effects of those variables whose impact was unknown.
Second, our aim was to be as broadly general as possible in
conclusions.  The programs from which we drew faults therefore had to be as realistic
as possible, and the faults had to be real, and plausible enough
to have happened in an environment where there was some standard of quality.

For these reasons, we selected the Java common library projects from
the Apache open-source project. Apache has a reputation of having a well maintained
standard of quality for their projects, and the commits that go into their
projects. They are also well known for having reasonably well tested projects.
By choosing projects from a single source, we minimize the variability due
to coding practices, development culture, and other possible confounding
factors. However, the tradeoff is that Apache common library projects may not
be representative of the broader software ecosystem. Further research is
necessary to ensure that our findings remain valid for the broader ecosystem.

\begin{table}
\centering
% latex table generated in R 3.2.3 by xtable 1.8-2 package
% Wed Jul  6 14:44:12 2016
\begin{tabular}{|r|l|r|r|r|r|}
  \hline
 & Projects & SLOC & TLOC & CPatches & Fails \\ 
  \hline
1 & commons-bcel & 30,175 & 3,155 & 148 &   6 \\ 
  2 & commons-beanutils & 11,640 & 21,665 &  63 &   5 \\ 
  3 & commons-cli & 2,665 & 3,768 &  71 &   5 \\ 
  4 & commons-codec & 6,599 & 11,026 & 179 &   4 \\ 
  5 & commons-collections & 27,820 & 32,913 & 333 &  16 \\ 
  6 & commons-compress & 18,746 & 13,496 & 430 &  65 \\ 
  7 & commons-configuration & 26,793 & 37,806 & 322 &  78 \\ 
  8 & commons-csv & 1,421 & 3,168 & 150 &   8 \\ 
  9 & commons-dbcp & 11,259 & 8,487 &  98 &  18 \\ 
  10 & commons-dbutils & 3,064 & 3,699 &  43 &   1 \\ 
  11 & commons-discovery & 2,320 & 268 & 171 &   1 \\ 
  12 & commons-exec & 1,757 & 1,601 &  90 &   5 \\ 
  13 & commons-fileupload & 2,389 & 1,946 & 129 &   8 \\ 
  14 & commons-imaging & 31,152 & 6,525 & 174 &   4 \\ 
  15 & commons-io & 9,813 & 17,968 & 177 &  18 \\ 
  16 & commons-jexl & 10,921 & 9,509 &  54 &  10 \\ 
  17 & commons-jxpath & 18,773 & 6,137 &  10 &   2 \\ 
  18 & commons-lang & 25,468 & 43,981 & 571 &  49 \\ 
  19 & commons-mail & 2,720 & 3,869 &  48 &   5 \\ 
  20 & commons-math & 84,809 & 89,336 & 954 & 142 \\ 
  21 & commons-net & 19,749 & 7,465 & 454 &  21 \\ 
  22 & commons-ognl & 13,139 & 6,873 & 190 &   3 \\ 
  23 & commons-pool & 5,242 & 8,042 & 149 &  12 \\ 
  24 & commons-scxml & 9,524 & 5,119 &  74 &   7 \\ 
  25 & commons-validator & 6,681 & 7,926 & 126 &  17 \\ 
   \hline
\end{tabular}

\caption{Apache Commons Libraries. SLOC is the program size in LOC, TLOC
is the test suite size in LOC, CPatches is the number of compiled patches,
and Fails is the number of test failures.}
\label{tbl:apache}
\end{table}

For our set of projects, we iterated through their commit
logs, and generated reverse patches for each commit. For each patch thus
created, we applied the patch on the latest repository, and removed any changes
to the test directory, thus ensuring that the test suite we tested with was
always the latest. Any patch that resulted in a compilation error was removed.
This resulted in a set of patches for each project that could be independently
applied. The complete test suite for the project was executed on each of the
patches left, and any patch that did not result in a test failure was
removed. The failed test cases that corresponded to each patch were thus
collected. At this point, we had a set of patches that introduce specific test
case failures.  The set of Apache projects, along with the set of reverse
patches thus found, are given in Table~\ref{tbl:apache}.

We conducted our remaining analysis in two parts. For the first part,
we generated patch pairs by joining together two random patches for
any given project. For the projects where the total number of unique pairs
were larger than $100$, we randomly sampled $100$ of the pairs produced.
After removing patch combinations that resulted in compilation errors, we had
1,126 patch combinations.
We evaluated the test suite of each project against the pair-patches thus
generated, and collected the test cases which failed against these.
\begin{table}
\centering
% latex table generated in R 3.2.3 by xtable 1.8-2 package
% Wed Jul  6 14:44:12 2016
\begin{tabular}{|l|r|}
  \hline
 & count \\ 
  \hline
All & 1,126 \\ 
  Coupled & 1,126 \\ 
  Subsuming &  56 \\ 
  Strongly Subsuming &   2 \\ 
  Strongly Subsuming and Coupled &   2 \\ 
  Weakly Subsuming and Coupled &  54 \\ 
  Weakly Subsuming and De-coupled &   0 \\ 
  Non Subsuming and De-coupled &   0 \\ 
  Non Subsuming and Coupled & 1,070 \\ 
   \hline
\end{tabular}

\caption{Higher Order Mutant Statistics}
\label{tbl:hom}
\end{table}
Adopting the terminology of Jia et al.~\cite{jia2009higher}, out of
1,126, we had 1,126 coupled higher order
mutants, and 56 subsuming mutants. Out of these,
there were only 2 strongly subsuming mutants. The
details are given in Table~\ref{tbl:hom}.

We tried to reduce the number of external variables further for the second
part, and chose a single large project --- \emph{Apache commons-math}.
We generated a set of combined patches by joining 2, 4, 8, 16, 32, and 64
patches at random, and evaluated the test suite for Apache commons-math against
each of these combined $k^{th}$ order patches. We removed all patches that
resulted in any compilation errors. This resulted in
342 patch combinations.

For both parts of our analysis, we generated two sets. The first set containing the unique
failures from the constituent faults in isolation, and the second, containing
failures from the joined patches.



\section{Analysis}
\label{sec:analysis}
There are two questions that we tackle here. The first investigates the
fraction of test cases that detect any of the constituent mutants that
also detect the combined mutant. That is, evaluates the following prediction
from the model:
``Given two faults, and the test cases killing each, (assuming a sufficiently
large \finput and \foutput, and ignoring the effects of strong interaction),
there is a high probability for the same test cases to kill the combined
fault.''

The second investigates the general coupling effect. Note that general coupling
effect does not distinguish between strong and weak interaction.
Since the general coupling ratio does not distinguish between strong and weak
interaction, this also serves as an evaluation of the strong interaction
between faults where inputs other than the original $i$ and $j$ -- that is
outside $i \cup j$ becomes faulty (where $i$ represents faulty inputs to $f$
due to faults in $h$, and $j$ represents faulty inputs to $f$ due to faults in $g$).

Indeed, we believe that strong interaction between different faults is
rarer than weak interaction. While there is no easy way to verify it, one may look
at the newer faults (new test failures) that are introduced by a combination of patches
when compared to the original patches as instances of strong fault interaction
(These are not the only instances, since some of the older tests could also
be failing on new behaviors. Further, some of the instances where new tests are
failing could also be due to too strict assertions failing only when multiple
conditions are satisfied. However, it is a reasonable proxy).

Our empirical evaluation does not require individual patches to be simple
faults. Rather, we are investigating the probability of fault masking
between \emph{any} two patches. Our theory suggests that irrespective of
whether the faults are complex or not, we can expect the same fault masking
probability.

\subsection{All Projects}
This section investigates fault pairs constructed from all projects.
\subsubsection{The Composite Fault Model}
Here, we try to answer the question: \emph{what percentage of test cases
detecting constituent faults can detect the complex faults?}

\begin{figure}[t]
\centering\textbf{All Projects: \couplingC}\par\medskip


{\centering \includegraphics[width=.99\linewidth]{../_R/figure/unnamed-chunk-4-1} 

}


\caption{The size of set of the test cases able to detect the faults when they were separate is in the \emph{x-axis},
and the \emph{subset of the same test cases} able to detect the combined fault is in the \emph{y-axis}. Different projects
are indicated by different colors.
The two projects that have a slightly below the mean relation between X and Y axis from other projects are \emph{commons-io} and \emph{commons-beanutils}.}
\label{fig:allfaults}
\end{figure}

Figure~\ref{fig:allfaults} plots the set of test cases able to detect
the faults when they were separate with the set of test cases able to
detect the combined fault.

To analyze the fraction of test cases expected to detect the combined mutant,
we evaluate the regression model given by:
\begin{equation}
\mu\{AfterT | BeforeT \} = \beta_0 + \beta_1 \times BeforeT
\label{eq:e1}
\end{equation}
where $BeforeT$ is the size of the test suite that includes all test cases that
can detect both faults separately, and $AfterT$ is the size of the test suite
which is a subset of $BeforeT$ that can detect the fault pair when combined.
We force $\beta_0$ to zero to account for the fact that if no test cases
detected the original mutant, then the question of their fraction does not
arise.
This linear regression model lets us predict the number of test fails for
combined faults from the test fails for separated faults.

We note that we are interested in $\beta_1$ for another purpose. $\beta_1$ is
also the \couplingC ratio $\kappa$. Thus this regression provides us with
a model for prediction, its goodness of fit ($R^2$), and also the \couplingC ratio.

\subsubsection{The General Coupling Model}
%It is investigated using a sample of the population of real faults, and their combinations
\begin{figure}[t]
\centering\textbf{All Projects: general coupling}\par\medskip


{\centering \includegraphics[width=.99\linewidth]{../_R/figure/unnamed-chunk-5-1} 

}


\caption{The size of the set of test cases able to detect
  the faults when they were separate is the \emph{x-axis},
  and the \emph{set of all test cases} able to detect the combined fault is in the \emph{y-axis}. Different projects
are indicated by different colors.
The three projects that have a slightly different relation between X and Y axis from other projects are \emph{commons-io} and \emph{commons-beanutils} (below the mean), and \emph{commons-dbcp} (above the mean).
}
\label{fig:allfaults1}
\end{figure}
Figure~\ref{fig:allfaults1} plots the general coupling of faults.
To answer this question, we evaluate the following regression model.
\begin{equation}
\mu\{NewT | BeforeT \} = \beta_0 + \beta_1 \times BeforeT
\label{eq:e2}
\end{equation}
where $BeforeT$ is the size of the test suite that includes all test cases that
can detect both faults separately, and $NewT$ is the size of the test suite
that can detect the fault pair when combined.
Note that we do not set $\beta_0 = 0$ here as the combined fault pair may be
detected by a new test case even if its constituents were not detected. In fact,
$\beta_0$ represents the complex faults that became detectable due to
interaction even though the constituent faults are not detectable.

However, if one wishes to investigate the general coupling ratio, we have to
investigate a simpler regression model, because the general coupling ratio does
not permit an intercept.
\begin{equation}
\mu\{NewT | BeforeT \} = \beta_1 \times BeforeT
\label{eq:e3}
\end{equation}

Here, similar to the previous section, $\beta_1$ corresponds to the general coupling ratio $C$.

\subsubsection{Strong fault interaction}

The incidence of strong fault interaction may be ascertained by the average number of new test
cases that failed for the combined patch. Note that this number is not exhaustive, as some of
the original test cases may fail for new faulty behavior too, even if the behavior is not same as that of the component faults.

\subsection{Apache Commons-math}

\subsubsection{The Composite Fault Model}
We try to answer the question \emph{what percentage of test cases detecting constituent faults can detect the complex faults?} for Apache Commons-math.

Figure~\ref{fig:mathfaults} plots the set of test cases able to detect
the faults when they were separate with the set of test cases able to
detect the combined fault. We rely on the regression given by Equation~\ref{eq:e1}
to answer this question.


\begin{figure}[t]
\centering\textbf{Commons-math: \couplingC}\par\medskip


{\centering \includegraphics[width=.99\linewidth]{../_R/figure/unnamed-chunk-6-1} 

}


\caption{The set of test cases able to detect the faults when they were separate is in the \emph{x-axis},
  and the \emph{subset of the same test cases} able to detect the combined fault is in the \emph{y-axis}.
The different number of patch combinations used are indicated by different colors.
}
\title{Commons-math: \couplingC}

\label{fig:mathfaults}
\end{figure}


\subsubsection{The General Coupling Model}
%semantic size of complex faults compared to their constituent faults?
Figure~\ref{fig:mathfaults1} plots the general coupling of faults for \emph{Apache commons math}.
We rely on the regressions given by Equation~\ref{eq:e2} and Equation~\ref{eq:e3} to answer this question.


\begin{figure}[t]
\centering\textbf{Commons-math: general coupling}\par\medskip


{\centering \includegraphics[width=.99\linewidth]{../_R/figure/unnamed-chunk-7-1} 

}


\caption{The set of test cases able to detect the faults when they were separate is in the \emph{x-axis},
  and the \emph{set of all test cases} able to detect the combined fault is in the \emph{y-axis}.
  The different number of patch combinations used are indicated by different colors.
}

\label{fig:mathfaults1}
\end{figure}

\subsubsection{Strong fault interaction}

The incidence of strong fault interaction may be ascertained by the average number of new test
cases that failed for the combined patch. The difference of note here is that the number of patches are larger, and hence the chances of strong interaction are correspondingly larger.


\section{Results}
\label{sec:results}

\subsection{All Projects}
\begin{table}
\centering
\resizebox{0.90\linewidth}{!}{
% latex table generated in R 3.2.3 by xtable 1.8-2 package
% Wed Jul  6 14:44:13 2016
\begin{tabular}{|r|r|r|r|r|}
  \hline
 & Estimate & Std. Error & t value & Pr($>$$|$t$|$) \\ 
  \hline
SeparateFaults & 0.9992 & 0.0005 & 2,116.13 & 0.0000 \\ 
   \hline
\end{tabular}

}
\caption{All projects regression for \kappaT. $R^2=$0.99975}
\label{tbl:allfraction}
\end{table}
The results for regression for Equation~\ref{eq:e1} for all projects is
given in Table~\ref{tbl:allfraction}.
The correlation between the dependent
and independent variable is 0.99975.
The \couplingC ratio
was found to be 0.99916.
\begin{table}
\centering
\resizebox{0.90\linewidth}{!}{
% latex table generated in R 3.2.3 by xtable 1.8-2 package
% Wed Jul  6 14:44:13 2016
\begin{tabular}{|r|r|r|r|r|}
  \hline
 & Estimate & Std. Error & t value & Pr($>$$|$t$|$) \\ 
  \hline
(Intercept) & -0.0399 & 0.0189 & -2.12 & 0.0343 \\ 
  SeparateFaults & 0.9997 & 0.0005 & 1,847.83 & 0.0000 \\ 
   \hline
\end{tabular}

}
\caption{All projects regression. $R^2=$0.99967}
\label{tbl:allsemantic}
\end{table}
\begin{table}
\centering
\resizebox{0.90\linewidth}{!}{
% latex table generated in R 3.2.3 by xtable 1.8-2 package
% Wed Jul  6 14:44:13 2016
\begin{tabular}{|r|r|r|r|r|}
  \hline
 & Estimate & Std. Error & t value & Pr($>$$|$t$|$) \\ 
  \hline
SeparateFaults & 0.9993 & 0.0005 & 1,939.82 & 0.0000 \\ 
   \hline
\end{tabular}

}
\caption{All projects regression for general coupling ratio. $R^2=$0.9997}
\label{tbl:xallsemantic}
\end{table}
The results for regression for Equation~\ref{eq:e2} for all projects is
given in Table~\ref{tbl:allsemantic}.
The correlation between the dependent
and independent variable is 0.99967.
The results for regression for Equation~\ref{eq:e3} for all projects is
given in Table~\ref{tbl:xallsemantic}.
The general coupling ratio
was found to be 0.99931.
Further, the mean number of faulty test cases that were not present in the
component faults were found to be 0.0417.
See Table~\ref{tbl:summaryall} for the summary.
\begin{table}
\centering
\resizebox{0.9\linewidth}{!}{
% latex table generated in R 3.2.3 by xtable 1.8-2 package
% Wed Jul  6 14:44:13 2016
\begin{tabular}{|r|r|r|r|r|}
  \hline
 & SeparateFaults & JoinedFaults & RemovedFaults & AddedFaults \\ 
  \hline
bcel & 27.73 & 27.73 & 0.00 & 0.00 \\ 
  beanutils & 4.80 & 1.60 & 3.20 & 0.00 \\ 
  cli & 7.60 & 7.60 & 0.00 & 0.00 \\ 
  codec & 2.50 & 2.50 & 0.00 & 0.00 \\ 
  collections & 16.49 & 16.49 & 0.00 & 0.00 \\ 
  compress & 11.60 & 11.60 & 0.00 & 0.00 \\ 
  configuration & 37.19 & 37.16 & 0.04 & 0.02 \\ 
  csv & 2.00 & 2.00 & 0.00 & 0.00 \\ 
  dbcp & 10.60 & 10.91 & 0.01 & 0.32 \\ 
  exec & 16.50 & 16.50 & 0.00 & 0.00 \\ 
  fileupload & 4.64 & 4.64 & 0.00 & 0.00 \\ 
  imaging & 6.50 & 6.50 & 0.00 & 0.00 \\ 
  io & 7.93 & 7.55 & 0.54 & 0.17 \\ 
  jexl & 3.58 & 3.56 & 0.02 & 0.00 \\ 
  jxpath & 3.00 & 3.00 & 0.00 & 0.00 \\ 
  lang & 4.46 & 4.46 & 0.00 & 0.00 \\ 
  mail & 2.30 & 2.30 & 0.00 & 0.00 \\ 
  math & 7.67 & 7.64 & 0.03 & 0.00 \\ 
  net & 4.13 & 4.13 & 0.00 & 0.00 \\ 
  ognl & 27.00 & 27.00 & 0.00 & 0.00 \\ 
  pool & 4.48 & 4.45 & 0.05 & 0.02 \\ 
  scxml & 36.62 & 36.62 & 0.00 & 0.00 \\ 
  validator & 3.07 & 3.06 & 0.01 & 0.00 \\ 
   \hline
\end{tabular}

}
\caption{Summary for all projects. The SeparateFaults column shows the number
of faults when the patches were separately analyzed, and JoinedFaults column
shows the number of faults when the patches were combined and analyzed.
RemovedFaults indicate the number of faults from SeparateFaults that were
removed in JoinedFaults, and AddedFaults indicate the number of newly added
faults that were not in SeparateFaults.}
\label{tbl:summaryall}
\end{table}

\subsection{Apache commons-math}
\begin{table}
\centering
\resizebox{0.90\linewidth}{!}{
% latex table generated in R 3.2.3 by xtable 1.8-2 package
% Wed Jul  6 14:44:13 2016
\begin{tabular}{|r|r|r|r|r|}
  \hline
 & Estimate & Std. Error & t value & Pr($>$$|$t$|$) \\ 
  \hline
SeparateFaults & 0.9896 & 0.0007 & 1,418.94 & 0.0000 \\ 
   \hline
\end{tabular}

}
\caption{Commons-math regression for \kappaT. $R^2=$0.99983}
\label{tbl:mathfraction}
\end{table}
The results for regression for Equation~\ref{eq:e1} for all projects is
given in Table~\ref{tbl:mathfraction}.
The correlation between the dependent
and independent variable is 0.99983.
The \couplingC ratio
was found to be 0.98956.
\begin{table}
\centering
\resizebox{0.90\linewidth}{!}{
% latex table generated in R 3.2.3 by xtable 1.8-2 package
% Wed Jul  6 14:44:13 2016
\begin{tabular}{|r|r|r|r|r|}
  \hline
 & Estimate & Std. Error & t value & Pr($>$$|$t$|$) \\ 
  \hline
(Intercept) & 0.0924 & 0.0482 & 1.92 & 0.0563 \\ 
  SeparateFaults & 0.9933 & 0.0009 & 1,090.92 & 0.0000 \\ 
   \hline
\end{tabular}

}
\caption{Commons-math regression. $R^2=$0.99971}
\label{tbl:mathsemantic}
\end{table}
\begin{table}
\centering
\resizebox{0.90\linewidth}{!}{
% latex table generated in R 3.2.3 by xtable 1.8-2 package
% Wed Jul  6 14:44:13 2016
\begin{tabular}{|r|r|r|r|r|}
  \hline
 & Estimate & Std. Error & t value & Pr($>$$|$t$|$) \\ 
  \hline
SeparateFaults & 0.9944 & 0.0007 & 1,401.55 & 0.0000 \\ 
   \hline
\end{tabular}

}
\caption{Commons-math regression for general coupling ratio. $R^2=$0.99983}
\label{tbl:xmathsemantic}
\end{table}
The results for regression for Equation~\ref{eq:e2} for commons-math is
given in Table~\ref{tbl:mathsemantic}.
The correlation between the dependent
and independent variable is 0.99971.
The results for regression for Equation~\ref{eq:e3} for commons-math is
given in Table~\ref{tbl:xmathsemantic}.
The general coupling ratio
was found to be 0.9944.
Further, the mean number of faulty test cases that were not present in the
component faults were found to be 0.137.
See Table~\ref{tbl:summarymath} for the summary.
% The interesting thing here
% is how the fault masking (RemovedFaults) and possible strong interaction
% (AddedFaults) increase along with the increase in number of patches.

\begin{table}
\centering
\resizebox{0.9\linewidth}{!}{
% latex table generated in R 3.2.3 by xtable 1.8-2 package
% Wed Jul  6 14:44:13 2016
\begin{tabular}{|r|r|r|r|r|}
  \hline
 & SeparateFaults & JoinedFaults & RemovedFaults & AddedFaults \\ 
  \hline
2 & 7.67 & 7.64 & 0.03 & 0.00 \\ 
  4 & 14.67 & 14.70 & 0.05 & 0.07 \\ 
  8 & 30.53 & 30.42 & 0.17 & 0.06 \\ 
  16 & 59.25 & 59.09 & 0.42 & 0.25 \\ 
  32 & 109.85 & 108.96 & 1.37 & 0.48 \\ 
  64 & 220.25 & 219.50 & 3.00 & 2.25 \\ 
   \hline
\end{tabular}

}
\caption{Summary for all Commons-math. The rownames indicate the number of
  patches involved.
  The SeparateFaults column shows the number
of faults when the patches were separately analyzed, and JoinedFaults column
shows the number of faults when the patches were combined and analyzed.
RemovedFaults indicate the number of faults from SeparateFaults that were
removed in JoinedFaults, and AddedFaults indicate the number of newly added
faults that were not in SeparateFaults.}
\label{tbl:summarymath}
\end{table}



\section{Discussion}
\label{sec:discussion}

Fault masking is one of the key concerns in software testing. The
\emph{coupling effect} hypothesis asserts that fault masking is rare,
and describes the general impact of interactions between
faults. Unfortunately, very little is known about the theory behind
fault coupling. We study the \emph{coupling effect} and fault masking
using a theoretical model, and evaluate our findings empirically.

Since the formal statement of \emph{coupling effect}
is vague and ambiguous, we provide a more concrete
definition (called the \efaultT to avoid confusion) ---
\textit{\cHypothesis}

We theoretically evaluate the \efaultT.
Our analysis shows that for any pair of separable faults, the \couplingC
effect exists. We find that \kappaT $\kappa = 1 - \frac{1}{n}$, where $n$
is the \foutput of the function being considered. We also show that the
consideration of the syntactical neighborhood does not have an adverse impact on
our result. Further, according to Wah, as system size increases, and
number of functions in the execution path nears the domain of function,
the coupling effect weakens exponentially. However, our results suggest
that the mean coupling ratio remains the same at $\frac{1}{n}$.

Why is our prediction on fault masking so important? The reason is
that basic testing relies on fault masking to a large extent. That is,
say you are unit testing a function with multiple faults, and some of
the faults are left undetected due to fault masking. Wah's analysis
suggests that when we integrate the units to a larger system, the faults
in larger system has a much larger (indeed exponential) tendency to self
correct, and avoid failure due to masking. Our analysis suggests that
even on larger systems composed of smaller systems, the rate of fault
masking remains the same.

We next empirically validated the \efaultT. Our subjects were numerous
and this allows us to make a strong general claim about the composite
coupling ratio $\kappa$, and also about the traditional coupling ratio $C$.

We also proposed the existence of strongly interacting faults, which cannot
be accounted for within the formal coupling theory. Our empirical analysis
(see Table~\ref{tbl:summaryall} and Table~\ref{tbl:summarymath}) indicates
that strong interaction is possibly rare, and is of similar frequency as
that of fault masking.

Examining Figure~\ref{fig:allfaults}, we note that while there is some
reduction in the combined faults for the faults with smaller semantic footprint
(as given by the number of test cases that failed for that fault) with respect
to the constituent faults, the difference vanishes when the size of the fault
increases. This same effect is also seen in Figure~\ref{fig:mathfaults}.
% It might also be noticed that for two projects (these were \emph{commons-io}
% and \emph{commons-beanutils}) very small constituent faults (with less than
% 10 test case failures) failures from complex faults are somewhat smaller than
% average, while for \emph{commons-dbcp}, the semantic size is slightly larger
% than average.
% (These were the only projects with statistically significant deviation in an anova).

The results for regression Equation~\ref{eq:e1} from all projects, and Apache
commons-math also suggests a similar observation --- that the test cases that
are able to detect a fault in isolation will with very high probability detect
the same fault when in combination with other faults.

A counter-intuitive result is suggested by the regression
(Equation~\ref{eq:e3}) comparing the semantic sizes of constituent faults and
combined faults. Table~\ref{tbl:xallsemantic} and Table~\ref{tbl:xmathsemantic}
suggests that the coefficient for $SeparateFaults$ is less than unity, which
means that when separate, the faults have a slightly larger semantic footprint
than when combined --- possibly due to fault interactions. However, we note that
the effect is not statistically significant for all projects. The coefficient
for separate faults for Table~\ref{tbl:xallsemantic} lies between
\{0.99831.0003\}, and the
coefficient for separate faults for Table~\ref{tbl:xmathsemantic} lies between
\{0.993000.99579\} --- 95\% confidence
interval, $p < 0.0001$.

Overall, our statistical analysis suggests that there is a very high
probability (between
\{0.998241.00009\} for all projects, and
\{0.988180.99093\} for math ---
95\% confidence interval with statistical significance $p < 0.0001$) that when
two faults are paired to produce a combined fault, any test cases that detected
either of the faults continue to detect the combined fault.

% Finally, we have already shown~\cite{gopinath2015howhard} that we can use
% sampling to approximate the mean of the entire population of faults.
Our results for Table~\ref{tbl:xallsemantic} suggests that between
\{0.99831.0003\} of complex
faults are caught (95\% confidence interval, $p < 0.0001$).
This is again confirmed by the deeper analysis of
\emph{Apache commons-math}, using larger size faults in Table~\ref{tbl:xmathsemantic}
which suggests that between
\{0.993000.99579\} fraction of complex
faults are caught (95\% confidence interval, $p < 0.0001$).
We note that this is the first confirmation of the \emph{general coupling effect}
(unlike the \emph{mutation coupling effect} which has been validated multiple times).
Why is validating general coupling effect important? We already know that faults
emulated by traditional mutants are only a subset of possible kinds of faults
(Just et al.~\cite{just2014mutants} found that up to 27\% of faults were inadequately
represented by mutants). Hence, it is important to verify \emph{general coupling
effect} using real faults so that our results are applicable for faults in
general, and especially for possible future mutation operators. Indeed,
the \emph{mutation coupling effect} has been
validated multiple times, and we do not attempt it again here.

How much difference is there between the traditional general coupling ratio $C$ and
the \kappaT $\kappa$? Our analysis finds that in general, there is less than a percentage
drop for the \kappaT from the traditional coupling ratio.


There is another point that is of interest in our study. Research in software
reliability, and software testing often relies on curated sets of
faults~\cite{just2014mutants,demillo1991on} that took manual effort to produce.
The paucity of such fault databases has held back research in software testing
to some extent. This also means that there is a high chance of skew either due to
manual bias, or due to skew in the projects thus selected.
Our technique is entirely automated. When a representative set of code
repositories are available, our methodology can be used to produce faults with
relatively less skew from external variables.



%External validity: Our formulation has been validated only using unit test
%suites. There exist a chance that system tests may have different behavior.

\section{Threats to Validity}
\label{sec:threats}

While we have taken every care to ensure that our results are unbiased, and
have tried to eliminate the effects of random noise, our results are subject
to the following threats to validity.

\subsection{Threats to Theoretical Analysis}
Our theoretical analysis relies on a main assumption, that all the faults of
a function have an equal probability of being chosen. We discussed the
effect of syntax on our model in Section~\ref{sec:theory}, where we noted
that syntax should not have a large impact so long as its impact is similar
across functions with expected output, and functions which do not have the
expected output. While there does not seem to be any reason to suspect
otherwise, and the results of empirical analysis seem to bear out our
theoretical result, there exists a possibility that this assumption is
unwarranted in general, or may depend on the particular program being
considered.

A second assumption that we made in the theoretical analysis is that the
\foutput of function being considered is large. To indicate the impact of
this assumption, for a function with binary \foutput (only true, and false are
allowed as output), there is $50\%$ chance of masking, and hence only 50\%
chance of coupling. However, the situation changes proportionally with
larger \foutput, with faults in a function with a possibility of output of even
$100$ unique values have $99\%$ chance of coupling. Hence,
we believe that the assumption of large \foutput is warranted on average.

Our theoretical analysis is incomplete for cases of more complex execution
paths where there may be cases where the solved templates for recursion and
iteration may not apply. This may also have an impact on the actual fault
masking for specific programs that do not fit into these templates.

Finally, we assumed that programs corresponded to mathematical functions when
taking that there were $n^m$ alternatives to a program $h \in : M \rightarrow N $.
However, in real world, syntax plays a much larger role, and there are
an infinite number of equivalent programs with exactly the same \finput and
\foutput.

We note that our empirical analysis does not verify the major difference
that we suggested from Wah's analysis --- that coupling effect does not weaken
with system size. This is not taken up here due to the significant investment
in finding and testing systems that are large enough. This is an avenue for
future work.

\subsection{Threats to Empirical Analysis}

\noindent\textbf{Threats due to sampling bias:} To ensure that our results were
representative of real world programs, with non-trivial faults, we opted for
Java projects from the Apache commons library. We used all projects
that we could build and complete testing. Further, we used all patches that
could be applied in isolation and resulted in a valid build, with at least
one test failure, which indicated that the patch fixed something in the
project. This however, implies that our sample of programs could be biased by
any factor that skews the kind of projects that the Apache community works on,
or their development practices. The representativeness of our sample patches
are also impacted by any factor that may skew the independent patches that
we extracted.

\noindent\textbf{Projects of differing maturity, and quality:} Since we used
real world projects from the Apache commons library, the maturity and quality
of these projects is representative of the real world.
However, not all projects have been in development for a long time, and hence
have a different distribution of faults from what is typically expected in an
industrial setting with a high standard of quality. To counter this threat
we evaluated Apache commons-math --- the largest, most mature, and highest
quality project which had 90\% statement coverage and 73.2\% mutation score for
our analysis for the second part. The results from Apache commons-math seem to
indicate that our experiment is relatively unbiased in this direction.



\section{Conclusion}
\label{sec:conclusion}

The \emph{coupling effect} hypothesis is a general theory of fault interaction,
and is used to quantify \emph{fault masking}. It also finds use in mutation
analysis. While there is interesting empirical evidence for the
\emph{coupling effect}, theoretical understanding is lacking. The extant theory
by Wah is too restrictive to be useful for real world systems.
We correct this limitation, and provide a stronger, modified version of
the theory called the \efaultT.

We provide a theory of fault coupling applicable to general
functions (with some restrictions). It incorporates the syntactical neighborhood
and clarify assumptions made. We show that even under real world conditions,
the \efaultT holds.

Our theoretical analysis suggests that the \faultT has
a high probability of occurring ($1 - \frac{1}{n}$, where $n$ is the \foutput
of the function under consideration) under the assumptions of total functions,
finite \finput, and separability of faults, \emph{irrespective of the size of
the system}.

Our empirical study provides validation, and an empirical approximation of
the \couplingC ratio
$\kappa$ (0.99), with 99\% of the test cases that detected a fault in
isolation continuing to detect it when it is combined with other faults.

Finally, the \emph{general coupling effect} has never been demonstrated for
real faults. Our empirical analysis also provides firm evidence in favor of the
\emph{general coupling effect} ($C = 99\%$) for real faults.


%\appendix
%\section{Appendix}
%<<lcacl, child='lambda-calc.Rnw'>>=
%@

%<<lcacl, child='combinatory-logic.Rnw'>>=
%@
% This is the text of the appendix, if you need one.

% \acks

% Acknowledgments, if needed.

% % We recommend abbrvnat bibliography style.

% \bibliographystyle{abbrvnat}

% % The bibliography should be embedded for final submission.

% \begin{thebibliography}{}
%     \softraggedright

%   \bibitem[Smith et~al.(2009)Smith, Jones]{smith02}
%     P. Q. Smith, and X. Y. Jones. ...reference text...

% \end{thebibliography}
%\balance
\bibliographystyle{acm}
\bibliography{paper}
\end{document}
