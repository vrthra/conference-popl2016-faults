%\usepackage{cite}
\usepackage{amssymb}
\usepackage{amsmath}
\newcommand{\cL}{{\cal L}}
\usepackage{placeins}
\usepackage{xspace}
\usepackage{boxedminipage}
\usepackage{tikz}

\usepackage{algpseudocode}
\usepackage{algorithm}
\usepackage{algorithmicx}
\newcommand{\theHalgorithm}{\arabic{algorithm}}

\usepackage[bookmarks=false, urlcolor=blue, linkcolor=blue, citecolor=blue, colorlinks=true]{hyperref} %\usepackage[center]{caption}
%\linespread{2}
\usepackage{comment}

\usepackage{listings}
\usepackage{float}
\floatstyle{plain}
\newfloat{program}{t}{lop}
\floatname{program}{Figure}
\usepackage{color}

\definecolor{dkgreen}{rgb}{0,0.6,0}
\definecolor{gray}{rgb}{0.5,0.5,0.5}
\definecolor{mauve}{rgb}{0.58,0,0.82}
\definecolor{lightgray}{gray}{0.75}

\newcommand\greybox[1]{%
  \vskip\baselineskip%
  \par\noindent\colorbox{lightgray}{%
    \begin{minipage}{0.99\columnwidth}#1\end{minipage}%
  }%
  \vskip\baselineskip%
}

\usepackage{blindtext}
\usepackage{framed}

\floatstyle{boxed}
\newfloat{infobox}{tbp}{ext}
\restylefloat*{infobox}
\floatname{infobox}{Note}

\floatstyle{ruled}
%\floatstyle{boxed}
\newfloat{mybox}{thp}{lop}
\floatname{mybox}{Box}

\newcommand{\faultT}{composite fault hypothesis\xspace}
\newcommand{\FaultT}{Composite fault hypothesis\xspace}
\newcommand{\efaultT}{\emph{\faultT}\xspace} %          composite fault hypothesis
\newcommand{\eFaultT}{\emph{\FaultT}\xspace} %          Composite fault hypothesis

\newcommand{\couplingC}{composite coupling\xspace}%          composite coupling
\newcommand{\CouplingC}{Composite coupling\xspace}%          Composite coupling
\newcommand{\kappaT}{\couplingC ratio\xspace}%          composite coupling ratio
\newcommand{\KappaT}{\CouplingC ratio\xspace}%          Composite coupling ratio

\newcommand{\thece}{the \emph{coupling effect}\xspace}
\newcommand{\Thece}{The \emph{coupling effect}\xspace}

\newcommand{\finput}{\emph{domain}\xspace}
\newcommand{\fInput}{\emph{Domain}\xspace}
\newcommand{\foutput}{\emph{co-domain}\xspace}
\newcommand{\fOutput}{\emph{Co-Domain}\xspace}

\newcommand{\oHypothesis} {
complex faults are coupled to simple faults in such
a way that a test data set that detects all simple faults in a program
will detect a high percentage of the complex faults.}

\newcommand{\cHypothesis}{
  Tests detecting a fault in isolation will (with high probability $\kappa$) continue to
detect the fault even when it occurs in combination with
other faults.\xspace
}%


\newcommand{\lcalc}{$\lambda$-calculus\xspace}
\newcommand{\clogic}{Combinatory-logic\xspace}
\newcommand{\qfunction}{\emph{q}-function\xspace}
\newcommand{\qfunctions}{\emph{q}-functions\xspace}
